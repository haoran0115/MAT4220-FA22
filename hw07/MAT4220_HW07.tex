\documentclass[twoside,11pt]{article}
\usepackage[left=1in, right=1in, top=1in, bottom=1in]{geometry}
\usepackage{amsmath}
\usepackage{amssymb}
\usepackage{amsfonts}
\usepackage{mathtools}
\usepackage{amsthm}
\usepackage{fancyhdr}
\usepackage{enumitem}
\usepackage{siunitx}
\usepackage{booktabs}
\usepackage[hidelinks]{hyperref}
\usepackage{sectsty}
\usepackage{mathrsfs} % mathscr
\usepackage{tikz}
\usepackage{pgfplots}
\usepackage{multicol}
\usepackage{listings}
\usepackage{soul}
\usepackage{braket}
\usepackage{esint}
% \usepackage{amsart}

% lmodern
\usepackage{lmodern}

% change mathcal shape
\usepackage[mathcal]{eucal}

% allow H option of figure
\usepackage{float}

% define math operators
\newcommand{\F}{\mathbb{F}}
\newcommand{\R}{\mathbb{R}}
\newcommand{\N}{\mathbb{N}}
\newcommand{\Z}{\mathbb{Z}}
\newcommand{\Q}{\mathbb{Q}}
\newcommand{\X}{\mathbb{Y}}
\renewcommand{\L}{\mathcal{L}}
% \renewcommand{\d}{\mathrm{d}}
\renewcommand*\d{\mathop{}\!\mathrm{d}}
\DeclareMathOperator*{\argmax}{arg\,max}
\DeclareMathOperator*{\argmin}{arg\,min}
\DeclareMathOperator{\im}{im}
\DeclareMathOperator{\id}{id}
\renewcommand{\mod}[1]{\ (\mathrm{mod}\ #1)}

% section font style
\sectionfont{\sffamily\Large}
\subsectionfont{\sffamily\normalsize}
\subsubsectionfont{\bf}

% line spreading and break
\hyphenpenalty=5000
\tolerance=20
\setlength{\parindent}{0em}
\setlength\parskip{0.5em}
\allowdisplaybreaks
\linespread{0.9}

% enumerate settings
% no break before enumerate
\setlist[enumerate]{itemsep=2pt,topsep=2pt}

% theorem
% latex theorem
% definition style
\theoremstyle{definition}
\newtheorem{theorem}{Theorem}[subsection]
\newtheorem{axiom}{Axiom}[section]
\newtheorem{definition}{Definition}[section]
\newtheorem{example}{Example}[section]
\newtheorem{question}{Question}[section]
\newtheorem{exercise}{Exercise}[section]
\newtheorem*{exercise*}{Exercise}
\newtheorem{lemma}{Lemma}[section]
\newtheorem{proposition}{Proposition}[section]
\newtheorem{corollary}{Corollary}[section]
\newtheorem*{theorem*}{Theorem}
\newtheorem{problem}{Problem}
% remark style
\theoremstyle{remark}
\newtheorem*{remark}{Remark}
\newtheorem*{solution}{Solution}
\newtheorem*{claim}{Claim}


% paragraph indent
\setlength{\parindent}{0em}
\setlength\parskip{0.5em}

\newcommand\Code{MAT4220 FA22}
\newcommand\Ass{HW07}
\newcommand\name{Haoran Sun}
\newcommand\mail{haoransun@link.cuhk.edu.cn}

\title{{\sffamily \Code \ \Ass}}
\author{\sffamily \name \ (\href{mailto:\mail}{\mail})}
\date{\sffamily \today}

\makeatletter
% \let\Title\@title
\let\theauthor\@author
\let\thedate\@date

\fancypagestyle{plain}{%
    \fancyhf{}
    \lhead{\sffamily \Ass}
    \rhead{\sffamily \name}
    \rfoot{\sffamily\thepage}

    % # 页脚自定义
    \fancyfoot[L]{
        \begin{minipage}[c]{0.06\textwidth}
            \includegraphics[height=7.5mm]{logo2.png}
        \end{minipage}
    }
}
\fancypagestyle{title}{%
    \fancyhf{}
    \renewcommand{\headrulewidth}{0pt}
    % \lhead{\Title}
    % \rhead{\theauthor}
    \rfoot{\sffamily\thepage}

    % # 页脚自定义
    \fancyfoot[L]{
        \begin{minipage}[c]{0.06\textwidth}
            \includegraphics[height=7.5mm]{logo2.png}
        \end{minipage}
    }
}
\fancyfootoffset[L]{0.3cm}

% re-define title format
\makeatletter
\renewcommand{\maketitle}{\bgroup\setlength{\parindent}{0pt}
\begin{flushleft}
  \textbf{\Large\@title}

  \@author
\end{flushleft}\egroup
}
\makeatother

\pagestyle{plain}

% lstlisting settings
\lstset{
    basicstyle=\linespread{0.7}\footnotesize,
    breaklines=true,
    basewidth=0.5em
}


\begin{document}
\maketitle
\thispagestyle{title}
% \begin{multicols*}{2}

% \begin{remark}
%     $V_\epsilon(x)$ is used to denote a $\epsilon$-neighborhood
%     \begin{align*}
%         V_\epsilon(x) = B_\epsilon(x)\setminus\{x\}
%     \end{align*}
% \end{remark}

\begin{problem}[P160 Q7]
The solution is in the form of
\begin{align*}
    u(r, \theta, \phi) &=
    \frac{c_1}{r} + c_2 + \frac{1}{6}r^2
\end{align*}
Applying the boundary condition where $u(a, \theta, \phi) = u(b, \theta, \phi) = 0$,
we can solve $c_1$ and $c_2$
\begin{align*}
    c_1 &= \frac{ab}{6}\frac{b^2-a^2}{b-a} = \frac{1}{6}ab(a+b) \quad
    c_2 = \frac{1}{6}\frac{b^3-a^3}{b-a} = \frac{1}{6}(a^2+ab+b^2)
\end{align*}
\end{problem}


\begin{problem}[P165 Q6]
Let $X(x)Y(y)Z(z)$ be a solution, then
\begin{align*}
    \Delta u = 0\Rightarrow
    \frac{X''}{X} + \frac{Y''}{Y} + \frac{Z''}{Z} = 0
\end{align*}
set boundary conditions
\begin{align*}
    u_x(0, y, z) = u_x(1, y, z) = u_y(x, 0, z) = u_y(x, 1, z)
    = u_z(x, y, 0) = 0
\end{align*}
We have
\begin{align*}
    X_m(x) = \cos m\pi x,\ Y_n(y) = \cos n\pi y
\end{align*}
Hence
\begin{align*}
    \frac{Z''}{Z} = (m^2+n^2)\pi^2 \Rightarrow
    Z(z) = A\cosh \sqrt{m^2+n^2}\pi z
\end{align*}
Let the solution be in the form
\begin{align*}
    u(x, y, z) &= 
    \frac{1}{4}A_{00} + 
    \frac{1}{2}\sum_{m=1}^\infty A_{m0}\cos m\pi x\cosh m\pi z + 
    \frac{1}{2}\sum_{n=1}^\infty A_{0n}\cos n\pi y\cosh m\pi z\\ &+
    \sum_{m=1}^\infty \sum_{n=1}^\infty A_{mn}\cos m\pi \cos n\pi y \cosh\sqrt{m^2+n^2}\pi z
\end{align*}
Hence
\begin{align*}
    u(x, y, 1) &= g(x, y) =
    \sum_{mn}A_{mn}\cos m\pi \cos n\pi y \cosh\sqrt{m^2+n^2}\pi\\
    \Rightarrow
    A_{mn} &= \frac{4}{\cosh\sqrt{m^2+n^2}\pi}\int_0^1\int_0^1\d x \d y
    \cos m\pi x \cos n\pi y g(x, y)
\end{align*}

\end{problem}


\begin{problem}[P172 Q2]
Since 
\begin{align*}
    u(r, \theta) &= \frac{1}{2}A_0 + \sum_{n=1}^\infty
    r^n(A_n\cos n\theta + B_n\sin n\theta)
\end{align*}
we have
\begin{align*}
    u(a, \theta) &= \frac{1}{2}A_0 + \sum_{n=1}^\infty
    a^n(A_n\cos n\theta + B_n\sin n\theta) = 1 + 3\sin\theta\\
    \Rightarrow A_0 &= 2,\ A_n = 0,\ B_1 = \frac{3}{a},\ B_2 = \cdots = 0
\end{align*}
which means $u(r, \theta) = 1 + 3r\sin\theta / a$.

\end{problem}



\begin{problem}[P176 Q4]
Let 
\begin{align*}
    u(r, \theta) &= \frac{1}{2}A_0 + \sum_{n=1}^\infty
    r^{-n}(A_n\cos n\phi + B_n\sin n\phi)\\
    A_n &= \frac{a^n}{\pi}\int_0^{2\pi}\d\phi h(\phi)\cos n\phi\\
    B_n &= \frac{a^n}{\pi}\int_0^{2\pi}\d\phi h(\phi)\sin n\phi
\end{align*}
Therefore
\begin{align*}
    u(r, \theta) &= \frac{1}{2}\int_0^{2\pi}\d\phi\ h(\phi) + 
    \sum_{n=1}^\infty (a/r)^{-n}\int_0^{2\pi} [h(\phi)\cos n\phi \cos n\theta
    + h(\phi)\sin n\phi\sin n\theta]\\
    &= \frac{1}{\pi}\int_0^{2\pi}\d \phi\ h(\phi)\left[ 
        \frac{1}{2} + \sum_{n=1}^\infty (a/r)^{-n}(\cos n\phi\cos n\theta
        + \sin n\phi \sin n\theta)
    \right]\\
    &= \frac{1}{\pi}\int_0^{2\pi}\d \phi\ h(\phi)\left[ 
        \frac{1}{2} + \sum_{n=1}^\infty
        (a/r)^{-n}\cos n(\phi-\theta)
    \right]\\
    &= \frac{1}{\pi}\int_0^{2\pi}\d \phi\ h(\phi)\left[ 
        \frac{1}{2} + \frac{1}{2}\sum_{n=1}^\infty
        (a/r)^{-n} e^{in\varphi} + 
        (a/r)^{-n} e^{-in\varphi}
    \right]\\
    &= \frac{1}{2\pi}\int_0^{2\pi}\d \phi\ h(\phi)\left[ 
        1 + \frac{ae^{i\varphi}}{r-ae^{i\varphi}} 
        + \frac{ae^{-i\varphi}}{r-ae^{-i\varphi}}
    \right]\\
    &= \frac{1}{2\pi}\int_0^{2\pi}\d \phi\ h(\phi) \frac{r^2-a^2}{r^2+a^2-2ar\cos(\theta-\phi)}
\end{align*}

\end{problem}


\begin{problem}[P176 Q13]
In this case, the eigenfunction $\Theta(\theta)$ and $R(r)$ have the form
\begin{align*}
    \Theta_n(\theta) &= \sin\frac{n\pi(\theta-\alpha)}{\beta - \alpha}\quad
    R_n(r) = A_n r^{\frac{n\pi}{\beta-\alpha}} + B_n r^{-\frac{n\pi}{\beta-\alpha}}
\end{align*}
Therefore
\begin{align*}
    u(r, \theta) &=
    \sum_{n=1}^\infty \left(
        A_n r^{\frac{n\pi}{\beta-\alpha}} + B_n r^{-\frac{n\pi}{\beta-\alpha}}
    \right)\sin\frac{n\pi(\theta-\alpha)}{\beta-\alpha}
    \\
    A_n &= \frac{2}{\beta-\alpha}\frac{1}{a^{\frac{2n\pi}{\beta-\alpha}}
    - b^{\frac{2n\pi}{\beta-\alpha}}}
    \int_\alpha^\beta \d\theta\sin\frac{n\pi(\theta-\alpha)}{\beta-\alpha}[
        a^{\frac{n\pi (\theta-\alpha)}{\beta-\alpha}}g(\theta) - 
        b^{\frac{n\pi (\theta-\alpha)}{\beta-\alpha}}h(\theta)
    ]
    \\
    B_n &= \frac{2}{\beta-\alpha}\frac{1}{a^{-\frac{2n\pi}{\beta-\alpha}}
    - b^{-\frac{2n\pi}{\beta-\alpha}}}
    \int_\alpha^\beta \d\theta\sin\frac{n\pi(\theta-\alpha)}{\beta-\alpha}[
        a^{-\frac{n\pi (\theta-\alpha)}{\beta-\alpha}}g(\theta) - 
        b^{-\frac{n\pi (\theta-\alpha)}{\beta-\alpha}}h(\theta)
    ]
\end{align*}
\end{problem}



% \end{multicols*}
\end{document}

