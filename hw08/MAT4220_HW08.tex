\documentclass[twoside,11pt]{article}
\usepackage[left=1in, right=1in, top=1in, bottom=1in]{geometry}
\usepackage{amsmath}
\usepackage{amssymb}
\usepackage{amsfonts}
\usepackage{mathtools}
\usepackage{amsthm}
\usepackage{fancyhdr}
\usepackage{enumitem}
\usepackage{siunitx}
\usepackage{booktabs}
\usepackage[hidelinks]{hyperref}
\usepackage{sectsty}
\usepackage{mathrsfs} % mathscr
\usepackage{tikz}
\usepackage{pgfplots}
\usepackage{multicol}
\usepackage{listings}
\usepackage{soul}
\usepackage{braket}
\usepackage{esint}
% \usepackage{amsart}

% lmodern
\usepackage{lmodern}

% change mathcal shape
\usepackage[mathcal]{eucal}

% allow H option of figure
\usepackage{float}

% define math operators
\newcommand{\F}{\mathbb{F}}
\newcommand{\R}{\mathbb{R}}
\newcommand{\N}{\mathbb{N}}
\newcommand{\Z}{\mathbb{Z}}
\newcommand{\Q}{\mathbb{Q}}
\newcommand{\X}{\mathbb{Y}}
\renewcommand{\L}{\mathcal{L}}
% \renewcommand{\d}{\mathrm{d}}
\renewcommand*\d{\mathop{}\!\mathrm{d}}
\DeclareMathOperator*{\argmax}{arg\,max}
\DeclareMathOperator*{\argmin}{arg\,min}
\DeclareMathOperator{\im}{im}
\DeclareMathOperator{\id}{id}
\renewcommand{\mod}[1]{\ (\mathrm{mod}\ #1)}

% section font style
\sectionfont{\sffamily\Large}
\subsectionfont{\sffamily\normalsize}
\subsubsectionfont{\bf}

% line spreading and break
\hyphenpenalty=5000
\tolerance=20
\setlength{\parindent}{0em}
\setlength\parskip{0.5em}
\allowdisplaybreaks
\linespread{0.9}

% enumerate settings
% no break before enumerate
\setlist[enumerate]{itemsep=2pt,topsep=2pt}

% theorem
% latex theorem
% definition style
\theoremstyle{definition}
\newtheorem{theorem}{Theorem}[subsection]
\newtheorem{axiom}{Axiom}[section]
\newtheorem{definition}{Definition}[section]
\newtheorem{example}{Example}[section]
\newtheorem{question}{Question}[section]
\newtheorem{exercise}{Exercise}[section]
\newtheorem*{exercise*}{Exercise}
\newtheorem{lemma}{Lemma}[section]
\newtheorem{proposition}{Proposition}[section]
\newtheorem{corollary}{Corollary}[section]
\newtheorem*{theorem*}{Theorem}
\newtheorem{problem}{Problem}
% remark style
\theoremstyle{remark}
\newtheorem*{remark}{Remark}
\newtheorem*{solution}{Solution}
\newtheorem*{claim}{Claim}


% paragraph indent
\setlength{\parindent}{0em}
\setlength\parskip{0.5em}

\newcommand\Code{MAT4220 FA22}
\newcommand\Ass{HW08}
\newcommand\name{Haoran Sun}
\newcommand\mail{haoransun@link.cuhk.edu.cn}

\title{{\sffamily \Code \ \Ass}}
\author{\sffamily \name \ (\href{mailto:\mail}{\mail})}
\date{\sffamily \today}

\makeatletter
% \let\Title\@title
\let\theauthor\@author
\let\thedate\@date

\fancypagestyle{plain}{%
    \fancyhf{}
    \lhead{\sffamily \Ass}
    \rhead{\sffamily \name}
    \rfoot{\sffamily\thepage}

    % # 页脚自定义
    \fancyfoot[L]{
        \begin{minipage}[c]{0.06\textwidth}
            \includegraphics[height=7.5mm]{logo2.png}
        \end{minipage}
    }
}
\fancypagestyle{title}{%
    \fancyhf{}
    \renewcommand{\headrulewidth}{0pt}
    % \lhead{\Title}
    % \rhead{\theauthor}
    \rfoot{\sffamily\thepage}

    % # 页脚自定义
    \fancyfoot[L]{
        \begin{minipage}[c]{0.06\textwidth}
            \includegraphics[height=7.5mm]{logo2.png}
        \end{minipage}
    }
}
\fancyfootoffset[L]{0.3cm}

% re-define title format
\makeatletter
\renewcommand{\maketitle}{\bgroup\setlength{\parindent}{0pt}
\begin{flushleft}
  \textbf{\Large\@title}

  \@author
\end{flushleft}\egroup
}
\makeatother

\pagestyle{plain}

% lstlisting settings
\lstset{
    basicstyle=\linespread{0.7}\footnotesize,
    breaklines=true,
    basewidth=0.5em
}


\begin{document}
\maketitle
\thispagestyle{title}
% \begin{multicols*}{2}

% \begin{remark}
%     $V_\epsilon(x)$ is used to denote a $\epsilon$-neighborhood
%     \begin{align*}
%         V_\epsilon(x) = B_\epsilon(x)\setminus\{x\}
%     \end{align*}
% \end{remark}

\begin{problem}[P184 Q5]
Consider any solutions satisfies the boundary condition
$w$ and $u$ where $\Delta u=0$.
Let $v=w-u$, hence $\partial v/\partial \mathbf{n} = 0$.
Then
\begin{align*}
    E[w] &= E[u+v] \\
    &= \frac{1}{2}\iiint_D\d\mathbf{x}\ 
    |\nabla u + \nabla v|^2 - \oiint_{\partial D}
    \d\sigma\ (u+v) (\nabla u + \nabla v)\cdot\mathbf{n}\\
    &= E[u] + E[v] + \iiint_D\d\mathbf{x}\ \nabla u\cdot\nabla v
    - \oiint_{\partial D}\d\sigma\ v\nabla u\cdot\mathbf{n}\\
    &= E[u] + E[v] + \oiint_{\partial D}\d\sigma\ v\nabla u\cdot\mathbf{n}
    - \iiint_D\d\mathbf{x}\ v\Delta u
    - \oiint_{\partial D}\d\sigma\ v\nabla u\cdot\mathbf{n}\\
    &= E[u] + E[v]
\end{align*}
Since
\begin{align*}
    E[v] &= \frac{1}{2}\iiint_D\d\mathbf{x}\ |\nabla v|^2
    - \oiint\d\sigma\ v\nabla v\cdot\mathbf{n}
    = \frac{1}{2}\iiint_D\d\mathbf{x}\ |\nabla v|^2
    \geq 0
\end{align*}
Hence $E[w]-E[u] = E[v]\geq 0$.
\end{problem}


% \begin{problem}[P184 Q6]
% {\color{red}
% \begin{remark}
% Although this problem is not included in the homework assignment,
% could you please help me to check if I am correct?
% \end{remark}}
% \begin{enumerate}[label=(\alph*)]
% \item Let $u$ and $v$ be two solutions.
% Then $w=u-v$ is a harmonic function with the following boundary condition
% \begin{align*}
%     \oiint_A\d\sigma\frac{\d w}{\d\mathbf{n}} 
%     &= 
%     \oiint_B\d\sigma\frac{\d w}{\d\mathbf{n}} = 0
% \end{align*}
% Hence according to the uniqueness of Neumann's problems we have 
% $w(\mathbf{x})=C$ for some constant $C$.
% Since $w(\mathbf{x})\rightarrow 0$ as $|\mathbf{x}|\rightarrow\infty$,
% we have $w(\mathbf{x})=u(\mathbf{x})-v(\mathbf{x}) = 0$, the solution
% is unique.

% \item Let $u(\mathbf{x})=a$ on $\partial A$ and $u(\mathbf{x})=b$ on $\partial B$.
% Then
% \begin{align*}
%     \oiint_{\partial A}\d\sigma\ u\frac{\partial u}{\partial \mathbf{n}}
%     &= a
%     \oiint_{\partial A}\d\sigma\frac{\partial u}{\partial \mathbf{n}} = Q>0
%     \Rightarrow a > 0
% \end{align*}
% Then easy to show that $u(\mathbf{x})$ takes its maximum on $\partial A$.
% Since the minimum must be obtained at the boundary, we know that $b$ on
% $\partial B$ is the minimum.


% \item In (b) we proved that $u$ takes its minimum $u=0$ on $\partial B$.
% Hence $u>0$ in $D$.


% \end{enumerate}
% \end{problem}


\begin{problem}[P184 Q7]
Define the operation
\begin{align*}
    (\nabla u_1, \nabla u_2) &= 
    \iiint_D\d\mathbf{x}\ \nabla u_1\cdot \nabla u_2
\end{align*}
Let
\begin{align*}
    \tilde{w}(\mathbf{x}) &= 
    w_0 + c_1w_1 + \cdots + c_nw_n
\end{align*}
let $c_0=1$, hence 
\begin{align*}
    E[\tilde{w}] &= 
    (\nabla \tilde{w}, \nabla\tilde{w})
    = \sum_{ij}c_i c_j(\nabla u_i, \nabla u_j)
\end{align*}
To minimize the energy, we should have $\partial E[\tilde{w}]/\partial c_i=0$,
which means
\begin{align*}
    \frac{\partial E[\tilde{w}]}{\partial c_i} &= 
    2c_i(\nabla w_i, \nabla w_i) + 2\sum_{i\neq j} c_j(\nabla w_i, \nabla w_j)
    = 2\sum_{j}c_j(\nabla w_i, \nabla w_j) = 0
\end{align*}
Hence we have a linear system with $n$ unknowns and $n$ equations
\begin{align*}
    \sum_{j=1}^n c_j(\nabla w_i, \nabla w_j) &= -(\nabla w_i, \nabla w_0)
\end{align*}
for $i=1,\dots,n$.
\end{problem}


\begin{problem}[P187 Q2]
Since
\begin{align*}
    -\frac{1}{4\pi}\iiint_D\d\mathbf{x}\frac{1}{r}\Delta\phi(\mathbf{x})
    &= 
    -\frac{1}{4\pi}\oiint_{\partial D}\d\sigma\frac{1}{r}\nabla\phi\cdot\mathbf{n}
    + \frac{1}{4\pi}\iiint_D\d\mathbf{x}\ \nabla\frac{1}{r}\cdot\nabla\phi(\mathbf{x})\\
    &= -\frac{1}{4\pi}\oiint_{\partial D}\d\sigma\frac{1}{r}\nabla\phi(\mathbf{x})\cdot\mathbf{n}
    + \frac{1}{4\pi}\oiint_{\partial D}\d\sigma \phi(\mathbf{x})\nabla\frac{1}{r}\cdot\mathbf{n}
    - \frac{1}{4\pi}\iiint_D\d\mathbf{x}\ \phi(\mathbf{x})\Delta\frac{1}{r}
\end{align*}
Let $D_\epsilon=B_R(\mathbf{0})\setminus B_\epsilon(\mathbf{0})$ for some $R>0$ large ($\phi$ vanish) 
and $\epsilon > 0$ small, since $\Delta 1/r=0$, then
\begin{align*}
    -\frac{1}{4\pi}\iiint_{D_\epsilon}\d\mathbf{x}\frac{1}{r}\Delta\phi(\mathbf{x})
    &= 
    - \frac{1}{4\pi}\oiint_{\partial B_\epsilon(\mathbf{0})}
    \d\sigma\frac{1}{r}\nabla\phi(\mathbf{x})\cdot\mathbf{n}
    + \frac{1}{4\pi}\oiint_{\partial B_\epsilon(\mathbf{0})}
    \d\sigma\phi(\mathbf{x})\nabla\frac{1}{r}\cdot\mathbf{n}\\
    &= 
    \epsilon\frac{1}{4\pi}\oiint_{\partial B_\epsilon(\mathbf{0})}
    \sin\theta\d\varphi\d\theta
    \frac{\partial }{\partial r}\phi(\mathbf{x})  
    + \frac{1}{4\pi}\oiint_{\partial B_\epsilon(\mathbf{0})}
    \sin\theta\d\varphi\d\theta \phi(\mathbf{x})\\
    &= \epsilon \frac{\partial}{\partial r}\phi(\mathbf{x}_1)
    + \phi(\mathbf{x}_2)
\end{align*}
for $\mathbf{x}_1,\mathbf{x}_2\in \partial B_\epsilon(\mathbf{0})$ according 
to the mean value theorem.
Hence, easy to show that
\begin{align*} 
    \lim_{\epsilon\rightarrow 0}
    -\frac{1}{4\pi}\iiint_{D_\epsilon}\d\mathbf{x}\frac{1}{r}\Delta\phi(\mathbf{x})
    &= 
    -\frac{1}{4\pi}\iiint_D\d\mathbf{x}\frac{1}{r}\Delta\phi(\mathbf{x})
    = 
    \phi(\mathbf{0})
\end{align*}

\end{problem}


\begin{problem}[P196 Q1]
Since $G''(x)=0$, we have $G(x)=Ax + b$ except $x_0$.
Then we have
\begin{align*}
    G(x) &= \begin{cases}
        Ax + B & x\in(0, x_0)\\
        Cx + D & x\in(x_0, l)
    \end{cases}
\end{align*}
Applying the boundary and continuity condition, we have
\begin{align*}
    B &= 0\\
    Cl + D &= 0\\
    Ax_0 + B &= Cx_0 + D
\end{align*}
Note that $H(x)=G(x) + |x-x_0|/2$ differentiable at $x_0$
\begin{align*}
    H(x) &= 
    \begin{cases}
        Ax + B + (x_0-x)/2 & x\in(0, x_0)\\
        Cx + D + (x-x_0)/2 & x\in(x_0, l)
    \end{cases}
    \Rightarrow
    A - 1/2 = C + 1/2
\end{align*}
Hence we have four equations, easy to solve that
\begin{align*}
    A = \frac{l-x_0}{x_0}\quad
    B = 0\quad
    C = -\frac{x_0}{l}\quad
    D = x_0
\end{align*}
Hence
\begin{align*}
    G(x, x_0) &= 
    \begin{cases}
        \frac{l-x_0}{l}x & x\in (0, x_0)\\
        \frac{l-x}{l}x_0 & x\in(x_0, l)
    \end{cases}
\end{align*}

\end{problem}


\begin{problem}[P196 Q6]\
\begin{enumerate}[label=(\alph*)]
\item The Green's function for the half plane is
\begin{align*}
    G(\mathbf{x}, \mathbf{x}_0) &= 
    \frac{1}{2\pi}(
        \log|\mathbf{x}-\mathbf{x}_0|
        - \log|\mathbf{x}-\mathbf{x}_0^*
    )
\end{align*}
where $\mathbf{x}_0=(x_0, y_0)$, $\mathbf{x}_0^*=(x_0, -y_0)$.

\item The solution is
\begin{align*}
    u(\mathbf{x}_0) &= 
    \int_{\partial D}\d\sigma\ h(\mathbf{x})\frac{\partial }{\partial\mathbf{n}}
    G(\mathbf{x}, \mathbf{x}_0)
    = -\int_{-\infty}^\infty\d x\ h(x)\frac{\partial}{\partial y}
    G[(x, 0), (x_0, y_0)]
\end{align*}
where
\begin{align*}
    \frac{\partial}{\partial y}G[(x, y), (x_0, y_0)] &= 
    \frac{1}{2\pi}\left[
        \frac{y-y_0}{(\mathbf{x}-\mathbf{x}_0)^2}
        -\frac{y+y_0}{(\mathbf{x}-\mathbf{x}_0^*)^2}
    \right]
\end{align*}

\item Plugin $h(\mathbf{x}) = 1$, we get
\begin{align*}
    u(x_0, y_0) &= 
    -\int_{-\infty}^\infty\d x\ \frac{1}{2\pi}
    \frac{0-y_0}{(x-x_0)^2+(0-y_0)^2}
    - \frac{0+y_0}{(x-x_0^2)+(0+y_0)^2}\\
    &= \frac{1}{\pi}\int_{-\infty}^\infty
    \d x\frac{y_0}{(x-x_0)^2 + y_0^2}\\
    &= \frac{1}{\pi}\int_{-\infty}^\infty\d u \frac{1}{u^2 + 1} = 1
\end{align*}

\end{enumerate}
\end{problem}



% \end{multicols*}
\end{document}

