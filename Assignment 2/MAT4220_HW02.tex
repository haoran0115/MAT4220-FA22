\documentclass[twoside,11pt]{article}
\usepackage[left=1in, right=1in, top=1in, bottom=1in]{geometry}
\usepackage{amsmath}
\usepackage{amssymb}
\usepackage{amsfonts}
\usepackage{mathtools}
\usepackage{amsthm}
\usepackage{fancyhdr}
\usepackage{enumitem}
\usepackage{siunitx}
\usepackage{booktabs}
\usepackage[hidelinks]{hyperref}
\usepackage{sectsty}
\usepackage{mathrsfs} % mathscr
\usepackage{tikz}
\usepackage{pgfplots}
\usepackage{multicol}
\usepackage{listings}
% \usepackage{amsart}

% change mathcal shape
\usepackage[mathcal]{eucal}

% allow H option of figure
\usepackage{float}

% define math operators
\newcommand{\F}{\mathbb{F}}
\newcommand{\R}{\mathbb{R}}
\newcommand{\N}{\mathbb{N}}
\newcommand{\Z}{\mathbb{Z}}
\newcommand{\Q}{\mathbb{Q}}
\newcommand{\X}{\mathbb{Y}}
\renewcommand{\L}{\mathcal{L}}
% \renewcommand{\d}{\mathrm{d}}
\renewcommand*\d{\mathop{}\!\mathrm{d}}
\DeclareMathOperator*{\argmax}{arg\,max}
\DeclareMathOperator*{\argmin}{arg\,min}
\DeclareMathOperator{\im}{im}
\DeclareMathOperator{\id}{id}
\renewcommand{\mod}[1]{\ (\mathrm{mod}\ #1)}

% section font style
\sectionfont{\sffamily\Large}
\subsectionfont{\sffamily\normalsize}
\subsubsectionfont{\bf}

% line spreading and break
\hyphenpenalty=5000
\tolerance=20
\setlength{\parindent}{0em}
\setlength\parskip{0.5em}
\allowdisplaybreaks
\linespread{0.9}

% theorem
% latex theorem
% definition style
\theoremstyle{definition}
\newtheorem{theorem}{Theorem}[subsection]
\newtheorem{axiom}{Axiom}[section]
\newtheorem{definition}{Definition}[section]
\newtheorem{example}{Example}[section]
\newtheorem{question}{Question}[section]
\newtheorem{exercise}{Exercise}[section]
\newtheorem*{exercise*}{Exercise}
\newtheorem{lemma}{Lemma}[section]
\newtheorem{proposition}{Proposition}[section]
\newtheorem{corollary}{Corollary}[section]
\newtheorem*{theorem*}{Theorem}
\newtheorem{problem}{Problem}
% remark style
\theoremstyle{remark}
\newtheorem*{remark}{Remark}
\newtheorem*{solution}{Solution}
\newtheorem*{claim}{Claim}


% paragraph indent
\setlength{\parindent}{0em}
\setlength\parskip{0.5em}

\newcommand\Code{MAT4220 FA22}
\newcommand\Ass{HW02}
\newcommand\name{Haoran Sun}
\newcommand\mail{haoransun@link.cuhk.edu.cn}

\title{{\sffamily \Code \ \Ass}}
\author{\sffamily \name \ (\href{mailto:\mail}{\mail})}
\date{\sffamily \today}

\makeatletter
% \let\Title\@title
\let\theauthor\@author
\let\thedate\@date

\fancypagestyle{plain}{%
    \fancyhf{}
    \lhead{\sffamily \Ass}
    \rhead{\sffamily \name}
    \rfoot{\sffamily\thepage}

    % # 页脚自定义
    \fancyfoot[L]{
        \begin{minipage}[c]{0.06\textwidth}
            \includegraphics[height=7.5mm]{logo2.png}
        \end{minipage}
    }
}
\fancypagestyle{title}{%
    \fancyhf{}
    \renewcommand{\headrulewidth}{0pt}
    % \lhead{\Title}
    % \rhead{\theauthor}
    \rfoot{\sffamily\thepage}

    % # 页脚自定义
    \fancyfoot[L]{
        \begin{minipage}[c]{0.06\textwidth}
            \includegraphics[height=7.5mm]{logo2.png}
        \end{minipage}
    }
}
\fancyfootoffset[L]{0.3cm}

% re-define title format
\makeatletter
\renewcommand{\maketitle}{\bgroup\setlength{\parindent}{0pt}
\begin{flushleft}
  \textbf{\Large\@title}

  \@author
\end{flushleft}\egroup
}
\makeatother

\pagestyle{plain}

% lstlisting settings
\lstset{
    basicstyle=\linespread{0.7}\footnotesize,
    breaklines=true,
    basewidth=0.5em
}


\begin{document}
\maketitle
\thispagestyle{title}
% \begin{multicols*}{2}

% \begin{remark}
%     $V_\epsilon(x)$ is used to denote a $\epsilon$-neighborhood
%     \begin{align*}
%         V_\epsilon(x) = B_\epsilon(x)\setminus\{x\}
%     \end{align*}
% \end{remark}

\begin{problem}[P38 Q2]
Using the formula for the wave equation
\begin{align*}
    u(x,t) &= \frac{1}{2}(\phi(ct+x) + \phi(-ct+x)) + \frac{1}{2c}\int_{-ct+x}^{ct+x} \phi(s)\d s\\
    &= \frac{1}{2}[\log(1+(ct+x)^2) + \log(1+(-ct+x)^2)] + \left. (4s+\frac{1}{2}s^2) \right |_{-ct+x}^{ct+x}\\
    &= \frac{1}{2}[\log(1+(ct+x)^2) + \log(1+(-ct+x)^2)] + 8ct + 2xct
\end{align*}
\end{problem}

\begin{problem}[P38 Q5]
Sketch:
\begin{figure}[H]
    \centering
    \resizebox{0.6\textwidth}{!}{%% Creator: Matplotlib, PGF backend
%%
%% To include the figure in your LaTeX document, write
%%   \input{<filename>.pgf}
%%
%% Make sure the required packages are loaded in your preamble
%%   \usepackage{pgf}
%%
%% Also ensure that all the required font packages are loaded; for instance,
%% the lmodern package is sometimes necessary when using math font.
%%   \usepackage{lmodern}
%%
%% Figures using additional raster images can only be included by \input if
%% they are in the same directory as the main LaTeX file. For loading figures
%% from other directories you can use the `import` package
%%   \usepackage{import}
%%
%% and then include the figures with
%%   \import{<path to file>}{<filename>.pgf}
%%
%% Matplotlib used the following preamble
%%   \usepackage{fontspec}
%%   \setmainfont{DejaVuSerif.ttf}[Path=\detokenize{D:/programs/miniconda3/envs/qm/lib/site-packages/matplotlib/mpl-data/fonts/ttf/}]
%%   \setsansfont{arial.ttf}[Path=\detokenize{C:/Windows/Fonts/}]
%%   \setmonofont{DejaVuSansMono.ttf}[Path=\detokenize{D:/programs/miniconda3/envs/qm/lib/site-packages/matplotlib/mpl-data/fonts/ttf/}]
%%
\begingroup%
\makeatletter%
\begin{pgfpicture}%
\pgfpathrectangle{\pgfpointorigin}{\pgfqpoint{8.000000in}{6.000000in}}%
\pgfusepath{use as bounding box, clip}%
\begin{pgfscope}%
\pgfsetbuttcap%
\pgfsetmiterjoin%
\definecolor{currentfill}{rgb}{1.000000,1.000000,1.000000}%
\pgfsetfillcolor{currentfill}%
\pgfsetlinewidth{0.000000pt}%
\definecolor{currentstroke}{rgb}{1.000000,1.000000,1.000000}%
\pgfsetstrokecolor{currentstroke}%
\pgfsetdash{}{0pt}%
\pgfpathmoveto{\pgfqpoint{0.000000in}{0.000000in}}%
\pgfpathlineto{\pgfqpoint{8.000000in}{0.000000in}}%
\pgfpathlineto{\pgfqpoint{8.000000in}{6.000000in}}%
\pgfpathlineto{\pgfqpoint{0.000000in}{6.000000in}}%
\pgfpathlineto{\pgfqpoint{0.000000in}{0.000000in}}%
\pgfpathclose%
\pgfusepath{fill}%
\end{pgfscope}%
\begin{pgfscope}%
\pgfsetbuttcap%
\pgfsetmiterjoin%
\definecolor{currentfill}{rgb}{1.000000,1.000000,1.000000}%
\pgfsetfillcolor{currentfill}%
\pgfsetlinewidth{0.000000pt}%
\definecolor{currentstroke}{rgb}{0.000000,0.000000,0.000000}%
\pgfsetstrokecolor{currentstroke}%
\pgfsetstrokeopacity{0.000000}%
\pgfsetdash{}{0pt}%
\pgfpathmoveto{\pgfqpoint{1.313713in}{0.792778in}}%
\pgfpathlineto{\pgfqpoint{7.760000in}{0.792778in}}%
\pgfpathlineto{\pgfqpoint{7.760000in}{5.760000in}}%
\pgfpathlineto{\pgfqpoint{1.313713in}{5.760000in}}%
\pgfpathlineto{\pgfqpoint{1.313713in}{0.792778in}}%
\pgfpathclose%
\pgfusepath{fill}%
\end{pgfscope}%
\begin{pgfscope}%
\pgfpathrectangle{\pgfqpoint{1.313713in}{0.792778in}}{\pgfqpoint{6.446287in}{4.967222in}}%
\pgfusepath{clip}%
\pgfsetroundcap%
\pgfsetroundjoin%
\pgfsetlinewidth{0.803000pt}%
\definecolor{currentstroke}{rgb}{0.800000,0.800000,0.800000}%
\pgfsetstrokecolor{currentstroke}%
\pgfsetdash{}{0pt}%
\pgfpathmoveto{\pgfqpoint{1.606726in}{0.792778in}}%
\pgfpathlineto{\pgfqpoint{1.606726in}{5.760000in}}%
\pgfusepath{stroke}%
\end{pgfscope}%
\begin{pgfscope}%
\definecolor{textcolor}{rgb}{0.150000,0.150000,0.150000}%
\pgfsetstrokecolor{textcolor}%
\pgfsetfillcolor{textcolor}%
\pgftext[x=1.606726in,y=0.695556in,,top]{\color{textcolor}\sffamily\fontsize{16.000000}{19.200000}\selectfont \ensuremath{-}6}%
\end{pgfscope}%
\begin{pgfscope}%
\pgfpathrectangle{\pgfqpoint{1.313713in}{0.792778in}}{\pgfqpoint{6.446287in}{4.967222in}}%
\pgfusepath{clip}%
\pgfsetroundcap%
\pgfsetroundjoin%
\pgfsetlinewidth{0.803000pt}%
\definecolor{currentstroke}{rgb}{0.800000,0.800000,0.800000}%
\pgfsetstrokecolor{currentstroke}%
\pgfsetdash{}{0pt}%
\pgfpathmoveto{\pgfqpoint{2.583436in}{0.792778in}}%
\pgfpathlineto{\pgfqpoint{2.583436in}{5.760000in}}%
\pgfusepath{stroke}%
\end{pgfscope}%
\begin{pgfscope}%
\definecolor{textcolor}{rgb}{0.150000,0.150000,0.150000}%
\pgfsetstrokecolor{textcolor}%
\pgfsetfillcolor{textcolor}%
\pgftext[x=2.583436in,y=0.695556in,,top]{\color{textcolor}\sffamily\fontsize{16.000000}{19.200000}\selectfont \ensuremath{-}4}%
\end{pgfscope}%
\begin{pgfscope}%
\pgfpathrectangle{\pgfqpoint{1.313713in}{0.792778in}}{\pgfqpoint{6.446287in}{4.967222in}}%
\pgfusepath{clip}%
\pgfsetroundcap%
\pgfsetroundjoin%
\pgfsetlinewidth{0.803000pt}%
\definecolor{currentstroke}{rgb}{0.800000,0.800000,0.800000}%
\pgfsetstrokecolor{currentstroke}%
\pgfsetdash{}{0pt}%
\pgfpathmoveto{\pgfqpoint{3.560147in}{0.792778in}}%
\pgfpathlineto{\pgfqpoint{3.560147in}{5.760000in}}%
\pgfusepath{stroke}%
\end{pgfscope}%
\begin{pgfscope}%
\definecolor{textcolor}{rgb}{0.150000,0.150000,0.150000}%
\pgfsetstrokecolor{textcolor}%
\pgfsetfillcolor{textcolor}%
\pgftext[x=3.560147in,y=0.695556in,,top]{\color{textcolor}\sffamily\fontsize{16.000000}{19.200000}\selectfont \ensuremath{-}2}%
\end{pgfscope}%
\begin{pgfscope}%
\pgfpathrectangle{\pgfqpoint{1.313713in}{0.792778in}}{\pgfqpoint{6.446287in}{4.967222in}}%
\pgfusepath{clip}%
\pgfsetroundcap%
\pgfsetroundjoin%
\pgfsetlinewidth{0.803000pt}%
\definecolor{currentstroke}{rgb}{0.800000,0.800000,0.800000}%
\pgfsetstrokecolor{currentstroke}%
\pgfsetdash{}{0pt}%
\pgfpathmoveto{\pgfqpoint{4.536857in}{0.792778in}}%
\pgfpathlineto{\pgfqpoint{4.536857in}{5.760000in}}%
\pgfusepath{stroke}%
\end{pgfscope}%
\begin{pgfscope}%
\definecolor{textcolor}{rgb}{0.150000,0.150000,0.150000}%
\pgfsetstrokecolor{textcolor}%
\pgfsetfillcolor{textcolor}%
\pgftext[x=4.536857in,y=0.695556in,,top]{\color{textcolor}\sffamily\fontsize{16.000000}{19.200000}\selectfont 0}%
\end{pgfscope}%
\begin{pgfscope}%
\pgfpathrectangle{\pgfqpoint{1.313713in}{0.792778in}}{\pgfqpoint{6.446287in}{4.967222in}}%
\pgfusepath{clip}%
\pgfsetroundcap%
\pgfsetroundjoin%
\pgfsetlinewidth{0.803000pt}%
\definecolor{currentstroke}{rgb}{0.800000,0.800000,0.800000}%
\pgfsetstrokecolor{currentstroke}%
\pgfsetdash{}{0pt}%
\pgfpathmoveto{\pgfqpoint{5.513567in}{0.792778in}}%
\pgfpathlineto{\pgfqpoint{5.513567in}{5.760000in}}%
\pgfusepath{stroke}%
\end{pgfscope}%
\begin{pgfscope}%
\definecolor{textcolor}{rgb}{0.150000,0.150000,0.150000}%
\pgfsetstrokecolor{textcolor}%
\pgfsetfillcolor{textcolor}%
\pgftext[x=5.513567in,y=0.695556in,,top]{\color{textcolor}\sffamily\fontsize{16.000000}{19.200000}\selectfont 2}%
\end{pgfscope}%
\begin{pgfscope}%
\pgfpathrectangle{\pgfqpoint{1.313713in}{0.792778in}}{\pgfqpoint{6.446287in}{4.967222in}}%
\pgfusepath{clip}%
\pgfsetroundcap%
\pgfsetroundjoin%
\pgfsetlinewidth{0.803000pt}%
\definecolor{currentstroke}{rgb}{0.800000,0.800000,0.800000}%
\pgfsetstrokecolor{currentstroke}%
\pgfsetdash{}{0pt}%
\pgfpathmoveto{\pgfqpoint{6.490277in}{0.792778in}}%
\pgfpathlineto{\pgfqpoint{6.490277in}{5.760000in}}%
\pgfusepath{stroke}%
\end{pgfscope}%
\begin{pgfscope}%
\definecolor{textcolor}{rgb}{0.150000,0.150000,0.150000}%
\pgfsetstrokecolor{textcolor}%
\pgfsetfillcolor{textcolor}%
\pgftext[x=6.490277in,y=0.695556in,,top]{\color{textcolor}\sffamily\fontsize{16.000000}{19.200000}\selectfont 4}%
\end{pgfscope}%
\begin{pgfscope}%
\pgfpathrectangle{\pgfqpoint{1.313713in}{0.792778in}}{\pgfqpoint{6.446287in}{4.967222in}}%
\pgfusepath{clip}%
\pgfsetroundcap%
\pgfsetroundjoin%
\pgfsetlinewidth{0.803000pt}%
\definecolor{currentstroke}{rgb}{0.800000,0.800000,0.800000}%
\pgfsetstrokecolor{currentstroke}%
\pgfsetdash{}{0pt}%
\pgfpathmoveto{\pgfqpoint{7.466987in}{0.792778in}}%
\pgfpathlineto{\pgfqpoint{7.466987in}{5.760000in}}%
\pgfusepath{stroke}%
\end{pgfscope}%
\begin{pgfscope}%
\definecolor{textcolor}{rgb}{0.150000,0.150000,0.150000}%
\pgfsetstrokecolor{textcolor}%
\pgfsetfillcolor{textcolor}%
\pgftext[x=7.466987in,y=0.695556in,,top]{\color{textcolor}\sffamily\fontsize{16.000000}{19.200000}\selectfont 6}%
\end{pgfscope}%
\begin{pgfscope}%
\definecolor{textcolor}{rgb}{0.150000,0.150000,0.150000}%
\pgfsetstrokecolor{textcolor}%
\pgfsetfillcolor{textcolor}%
\pgftext[x=4.536857in,y=0.436767in,,top]{\color{textcolor}\sffamily\fontsize{16.000000}{19.200000}\selectfont \(\displaystyle x\)}%
\end{pgfscope}%
\begin{pgfscope}%
\pgfpathrectangle{\pgfqpoint{1.313713in}{0.792778in}}{\pgfqpoint{6.446287in}{4.967222in}}%
\pgfusepath{clip}%
\pgfsetroundcap%
\pgfsetroundjoin%
\pgfsetlinewidth{0.803000pt}%
\definecolor{currentstroke}{rgb}{0.800000,0.800000,0.800000}%
\pgfsetstrokecolor{currentstroke}%
\pgfsetdash{}{0pt}%
\pgfpathmoveto{\pgfqpoint{1.313713in}{1.147579in}}%
\pgfpathlineto{\pgfqpoint{7.760000in}{1.147579in}}%
\pgfusepath{stroke}%
\end{pgfscope}%
\begin{pgfscope}%
\definecolor{textcolor}{rgb}{0.150000,0.150000,0.150000}%
\pgfsetstrokecolor{textcolor}%
\pgfsetfillcolor{textcolor}%
\pgftext[x=0.511266in, y=1.057687in, left, base]{\color{textcolor}\sffamily\fontsize{16.000000}{19.200000}\selectfont 0.0 \(\displaystyle a/c\)}%
\end{pgfscope}%
\begin{pgfscope}%
\pgfpathrectangle{\pgfqpoint{1.313713in}{0.792778in}}{\pgfqpoint{6.446287in}{4.967222in}}%
\pgfusepath{clip}%
\pgfsetroundcap%
\pgfsetroundjoin%
\pgfsetlinewidth{0.803000pt}%
\definecolor{currentstroke}{rgb}{0.800000,0.800000,0.800000}%
\pgfsetstrokecolor{currentstroke}%
\pgfsetdash{}{0pt}%
\pgfpathmoveto{\pgfqpoint{1.313713in}{1.857183in}}%
\pgfpathlineto{\pgfqpoint{7.760000in}{1.857183in}}%
\pgfusepath{stroke}%
\end{pgfscope}%
\begin{pgfscope}%
\definecolor{textcolor}{rgb}{0.150000,0.150000,0.150000}%
\pgfsetstrokecolor{textcolor}%
\pgfsetfillcolor{textcolor}%
\pgftext[x=0.511266in, y=1.767291in, left, base]{\color{textcolor}\sffamily\fontsize{16.000000}{19.200000}\selectfont 0.2 \(\displaystyle a/c\)}%
\end{pgfscope}%
\begin{pgfscope}%
\pgfpathrectangle{\pgfqpoint{1.313713in}{0.792778in}}{\pgfqpoint{6.446287in}{4.967222in}}%
\pgfusepath{clip}%
\pgfsetroundcap%
\pgfsetroundjoin%
\pgfsetlinewidth{0.803000pt}%
\definecolor{currentstroke}{rgb}{0.800000,0.800000,0.800000}%
\pgfsetstrokecolor{currentstroke}%
\pgfsetdash{}{0pt}%
\pgfpathmoveto{\pgfqpoint{1.313713in}{2.566786in}}%
\pgfpathlineto{\pgfqpoint{7.760000in}{2.566786in}}%
\pgfusepath{stroke}%
\end{pgfscope}%
\begin{pgfscope}%
\definecolor{textcolor}{rgb}{0.150000,0.150000,0.150000}%
\pgfsetstrokecolor{textcolor}%
\pgfsetfillcolor{textcolor}%
\pgftext[x=0.511266in, y=2.476894in, left, base]{\color{textcolor}\sffamily\fontsize{16.000000}{19.200000}\selectfont 0.4 \(\displaystyle a/c\)}%
\end{pgfscope}%
\begin{pgfscope}%
\pgfpathrectangle{\pgfqpoint{1.313713in}{0.792778in}}{\pgfqpoint{6.446287in}{4.967222in}}%
\pgfusepath{clip}%
\pgfsetroundcap%
\pgfsetroundjoin%
\pgfsetlinewidth{0.803000pt}%
\definecolor{currentstroke}{rgb}{0.800000,0.800000,0.800000}%
\pgfsetstrokecolor{currentstroke}%
\pgfsetdash{}{0pt}%
\pgfpathmoveto{\pgfqpoint{1.313713in}{3.276389in}}%
\pgfpathlineto{\pgfqpoint{7.760000in}{3.276389in}}%
\pgfusepath{stroke}%
\end{pgfscope}%
\begin{pgfscope}%
\definecolor{textcolor}{rgb}{0.150000,0.150000,0.150000}%
\pgfsetstrokecolor{textcolor}%
\pgfsetfillcolor{textcolor}%
\pgftext[x=0.511266in, y=3.186497in, left, base]{\color{textcolor}\sffamily\fontsize{16.000000}{19.200000}\selectfont 0.6 \(\displaystyle a/c\)}%
\end{pgfscope}%
\begin{pgfscope}%
\pgfpathrectangle{\pgfqpoint{1.313713in}{0.792778in}}{\pgfqpoint{6.446287in}{4.967222in}}%
\pgfusepath{clip}%
\pgfsetroundcap%
\pgfsetroundjoin%
\pgfsetlinewidth{0.803000pt}%
\definecolor{currentstroke}{rgb}{0.800000,0.800000,0.800000}%
\pgfsetstrokecolor{currentstroke}%
\pgfsetdash{}{0pt}%
\pgfpathmoveto{\pgfqpoint{1.313713in}{3.985992in}}%
\pgfpathlineto{\pgfqpoint{7.760000in}{3.985992in}}%
\pgfusepath{stroke}%
\end{pgfscope}%
\begin{pgfscope}%
\definecolor{textcolor}{rgb}{0.150000,0.150000,0.150000}%
\pgfsetstrokecolor{textcolor}%
\pgfsetfillcolor{textcolor}%
\pgftext[x=0.511266in, y=3.896100in, left, base]{\color{textcolor}\sffamily\fontsize{16.000000}{19.200000}\selectfont 0.8 \(\displaystyle a/c\)}%
\end{pgfscope}%
\begin{pgfscope}%
\pgfpathrectangle{\pgfqpoint{1.313713in}{0.792778in}}{\pgfqpoint{6.446287in}{4.967222in}}%
\pgfusepath{clip}%
\pgfsetroundcap%
\pgfsetroundjoin%
\pgfsetlinewidth{0.803000pt}%
\definecolor{currentstroke}{rgb}{0.800000,0.800000,0.800000}%
\pgfsetstrokecolor{currentstroke}%
\pgfsetdash{}{0pt}%
\pgfpathmoveto{\pgfqpoint{1.313713in}{4.695595in}}%
\pgfpathlineto{\pgfqpoint{7.760000in}{4.695595in}}%
\pgfusepath{stroke}%
\end{pgfscope}%
\begin{pgfscope}%
\definecolor{textcolor}{rgb}{0.150000,0.150000,0.150000}%
\pgfsetstrokecolor{textcolor}%
\pgfsetfillcolor{textcolor}%
\pgftext[x=0.511266in, y=4.605703in, left, base]{\color{textcolor}\sffamily\fontsize{16.000000}{19.200000}\selectfont 1.0 \(\displaystyle a/c\)}%
\end{pgfscope}%
\begin{pgfscope}%
\pgfpathrectangle{\pgfqpoint{1.313713in}{0.792778in}}{\pgfqpoint{6.446287in}{4.967222in}}%
\pgfusepath{clip}%
\pgfsetroundcap%
\pgfsetroundjoin%
\pgfsetlinewidth{0.803000pt}%
\definecolor{currentstroke}{rgb}{0.800000,0.800000,0.800000}%
\pgfsetstrokecolor{currentstroke}%
\pgfsetdash{}{0pt}%
\pgfpathmoveto{\pgfqpoint{1.313713in}{5.405198in}}%
\pgfpathlineto{\pgfqpoint{7.760000in}{5.405198in}}%
\pgfusepath{stroke}%
\end{pgfscope}%
\begin{pgfscope}%
\definecolor{textcolor}{rgb}{0.150000,0.150000,0.150000}%
\pgfsetstrokecolor{textcolor}%
\pgfsetfillcolor{textcolor}%
\pgftext[x=0.511266in, y=5.315307in, left, base]{\color{textcolor}\sffamily\fontsize{16.000000}{19.200000}\selectfont 1.2 \(\displaystyle a/c\)}%
\end{pgfscope}%
\begin{pgfscope}%
\definecolor{textcolor}{rgb}{0.150000,0.150000,0.150000}%
\pgfsetstrokecolor{textcolor}%
\pgfsetfillcolor{textcolor}%
\pgftext[x=0.455710in,y=3.276389in,,bottom,rotate=90.000000]{\color{textcolor}\sffamily\fontsize{16.000000}{19.200000}\selectfont \(\displaystyle u\)}%
\end{pgfscope}%
\begin{pgfscope}%
\pgfpathrectangle{\pgfqpoint{1.313713in}{0.792778in}}{\pgfqpoint{6.446287in}{4.967222in}}%
\pgfusepath{clip}%
\pgfsetroundcap%
\pgfsetroundjoin%
\pgfsetlinewidth{1.505625pt}%
\definecolor{currentstroke}{rgb}{0.121569,0.466667,0.705882}%
\pgfsetstrokecolor{currentstroke}%
\pgfsetdash{}{0pt}%
\pgfpathmoveto{\pgfqpoint{3.804324in}{1.147579in}}%
\pgfpathlineto{\pgfqpoint{4.048502in}{2.921587in}}%
\pgfpathlineto{\pgfqpoint{5.025212in}{2.921587in}}%
\pgfpathlineto{\pgfqpoint{5.269389in}{1.147579in}}%
\pgfusepath{stroke}%
\end{pgfscope}%
\begin{pgfscope}%
\pgfpathrectangle{\pgfqpoint{1.313713in}{0.792778in}}{\pgfqpoint{6.446287in}{4.967222in}}%
\pgfusepath{clip}%
\pgfsetroundcap%
\pgfsetroundjoin%
\pgfsetlinewidth{1.505625pt}%
\definecolor{currentstroke}{rgb}{1.000000,0.498039,0.054902}%
\pgfsetstrokecolor{currentstroke}%
\pgfsetdash{}{0pt}%
\pgfpathmoveto{\pgfqpoint{3.560147in}{1.147579in}}%
\pgfpathlineto{\pgfqpoint{4.536857in}{4.695595in}}%
\pgfpathlineto{\pgfqpoint{5.513567in}{1.147579in}}%
\pgfusepath{stroke}%
\end{pgfscope}%
\begin{pgfscope}%
\pgfpathrectangle{\pgfqpoint{1.313713in}{0.792778in}}{\pgfqpoint{6.446287in}{4.967222in}}%
\pgfusepath{clip}%
\pgfsetroundcap%
\pgfsetroundjoin%
\pgfsetlinewidth{1.505625pt}%
\definecolor{currentstroke}{rgb}{0.172549,0.627451,0.172549}%
\pgfsetstrokecolor{currentstroke}%
\pgfsetdash{}{0pt}%
\pgfpathmoveto{\pgfqpoint{3.315969in}{1.147579in}}%
\pgfpathlineto{\pgfqpoint{4.292679in}{4.695595in}}%
\pgfpathlineto{\pgfqpoint{4.781034in}{4.695595in}}%
\pgfpathlineto{\pgfqpoint{5.757744in}{1.147579in}}%
\pgfusepath{stroke}%
\end{pgfscope}%
\begin{pgfscope}%
\pgfpathrectangle{\pgfqpoint{1.313713in}{0.792778in}}{\pgfqpoint{6.446287in}{4.967222in}}%
\pgfusepath{clip}%
\pgfsetroundcap%
\pgfsetroundjoin%
\pgfsetlinewidth{1.505625pt}%
\definecolor{currentstroke}{rgb}{0.839216,0.152941,0.156863}%
\pgfsetstrokecolor{currentstroke}%
\pgfsetdash{}{0pt}%
\pgfpathmoveto{\pgfqpoint{3.071791in}{1.147579in}}%
\pgfpathlineto{\pgfqpoint{4.048502in}{4.695595in}}%
\pgfpathlineto{\pgfqpoint{5.025212in}{4.695595in}}%
\pgfpathlineto{\pgfqpoint{6.001922in}{1.147579in}}%
\pgfusepath{stroke}%
\end{pgfscope}%
\begin{pgfscope}%
\pgfpathrectangle{\pgfqpoint{1.313713in}{0.792778in}}{\pgfqpoint{6.446287in}{4.967222in}}%
\pgfusepath{clip}%
\pgfsetroundcap%
\pgfsetroundjoin%
\pgfsetlinewidth{1.505625pt}%
\definecolor{currentstroke}{rgb}{0.580392,0.403922,0.741176}%
\pgfsetstrokecolor{currentstroke}%
\pgfsetdash{}{0pt}%
\pgfpathmoveto{\pgfqpoint{1.606726in}{1.147579in}}%
\pgfpathlineto{\pgfqpoint{2.583436in}{4.695595in}}%
\pgfpathlineto{\pgfqpoint{6.490277in}{4.695595in}}%
\pgfpathlineto{\pgfqpoint{7.466987in}{1.147579in}}%
\pgfusepath{stroke}%
\end{pgfscope}%
\begin{pgfscope}%
\pgfsetrectcap%
\pgfsetmiterjoin%
\pgfsetlinewidth{0.803000pt}%
\definecolor{currentstroke}{rgb}{0.800000,0.800000,0.800000}%
\pgfsetstrokecolor{currentstroke}%
\pgfsetdash{}{0pt}%
\pgfpathmoveto{\pgfqpoint{1.313713in}{0.792778in}}%
\pgfpathlineto{\pgfqpoint{1.313713in}{5.760000in}}%
\pgfusepath{stroke}%
\end{pgfscope}%
\begin{pgfscope}%
\pgfsetrectcap%
\pgfsetmiterjoin%
\pgfsetlinewidth{0.803000pt}%
\definecolor{currentstroke}{rgb}{0.800000,0.800000,0.800000}%
\pgfsetstrokecolor{currentstroke}%
\pgfsetdash{}{0pt}%
\pgfpathmoveto{\pgfqpoint{7.760000in}{0.792778in}}%
\pgfpathlineto{\pgfqpoint{7.760000in}{5.760000in}}%
\pgfusepath{stroke}%
\end{pgfscope}%
\begin{pgfscope}%
\pgfsetrectcap%
\pgfsetmiterjoin%
\pgfsetlinewidth{0.803000pt}%
\definecolor{currentstroke}{rgb}{0.800000,0.800000,0.800000}%
\pgfsetstrokecolor{currentstroke}%
\pgfsetdash{}{0pt}%
\pgfpathmoveto{\pgfqpoint{1.313713in}{0.792778in}}%
\pgfpathlineto{\pgfqpoint{7.760000in}{0.792778in}}%
\pgfusepath{stroke}%
\end{pgfscope}%
\begin{pgfscope}%
\pgfsetrectcap%
\pgfsetmiterjoin%
\pgfsetlinewidth{0.803000pt}%
\definecolor{currentstroke}{rgb}{0.800000,0.800000,0.800000}%
\pgfsetstrokecolor{currentstroke}%
\pgfsetdash{}{0pt}%
\pgfpathmoveto{\pgfqpoint{1.313713in}{5.760000in}}%
\pgfpathlineto{\pgfqpoint{7.760000in}{5.760000in}}%
\pgfusepath{stroke}%
\end{pgfscope}%
\begin{pgfscope}%
\pgfsetbuttcap%
\pgfsetmiterjoin%
\definecolor{currentfill}{rgb}{1.000000,1.000000,1.000000}%
\pgfsetfillcolor{currentfill}%
\pgfsetfillopacity{0.800000}%
\pgfsetlinewidth{1.003750pt}%
\definecolor{currentstroke}{rgb}{0.800000,0.800000,0.800000}%
\pgfsetstrokecolor{currentstroke}%
\pgfsetstrokeopacity{0.800000}%
\pgfsetdash{}{0pt}%
\pgfpathmoveto{\pgfqpoint{5.947678in}{3.828752in}}%
\pgfpathlineto{\pgfqpoint{7.604444in}{3.828752in}}%
\pgfpathquadraticcurveto{\pgfqpoint{7.648889in}{3.828752in}}{\pgfqpoint{7.648889in}{3.873197in}}%
\pgfpathlineto{\pgfqpoint{7.648889in}{5.604444in}}%
\pgfpathquadraticcurveto{\pgfqpoint{7.648889in}{5.648889in}}{\pgfqpoint{7.604444in}{5.648889in}}%
\pgfpathlineto{\pgfqpoint{5.947678in}{5.648889in}}%
\pgfpathquadraticcurveto{\pgfqpoint{5.903234in}{5.648889in}}{\pgfqpoint{5.903234in}{5.604444in}}%
\pgfpathlineto{\pgfqpoint{5.903234in}{3.873197in}}%
\pgfpathquadraticcurveto{\pgfqpoint{5.903234in}{3.828752in}}{\pgfqpoint{5.947678in}{3.828752in}}%
\pgfpathlineto{\pgfqpoint{5.947678in}{3.828752in}}%
\pgfpathclose%
\pgfusepath{stroke,fill}%
\end{pgfscope}%
\begin{pgfscope}%
\pgfsetroundcap%
\pgfsetroundjoin%
\pgfsetlinewidth{1.505625pt}%
\definecolor{currentstroke}{rgb}{0.121569,0.466667,0.705882}%
\pgfsetstrokecolor{currentstroke}%
\pgfsetdash{}{0pt}%
\pgfpathmoveto{\pgfqpoint{5.992123in}{5.457994in}}%
\pgfpathlineto{\pgfqpoint{6.214345in}{5.457994in}}%
\pgfpathlineto{\pgfqpoint{6.436567in}{5.457994in}}%
\pgfusepath{stroke}%
\end{pgfscope}%
\begin{pgfscope}%
\definecolor{textcolor}{rgb}{0.150000,0.150000,0.150000}%
\pgfsetstrokecolor{textcolor}%
\pgfsetfillcolor{textcolor}%
\pgftext[x=6.614345in,y=5.380216in,left,base]{\color{textcolor}\sffamily\fontsize{16.000000}{19.200000}\selectfont \(\displaystyle t=a/2c\)}%
\end{pgfscope}%
\begin{pgfscope}%
\pgfsetroundcap%
\pgfsetroundjoin%
\pgfsetlinewidth{1.505625pt}%
\definecolor{currentstroke}{rgb}{1.000000,0.498039,0.054902}%
\pgfsetstrokecolor{currentstroke}%
\pgfsetdash{}{0pt}%
\pgfpathmoveto{\pgfqpoint{5.992123in}{5.107300in}}%
\pgfpathlineto{\pgfqpoint{6.214345in}{5.107300in}}%
\pgfpathlineto{\pgfqpoint{6.436567in}{5.107300in}}%
\pgfusepath{stroke}%
\end{pgfscope}%
\begin{pgfscope}%
\definecolor{textcolor}{rgb}{0.150000,0.150000,0.150000}%
\pgfsetstrokecolor{textcolor}%
\pgfsetfillcolor{textcolor}%
\pgftext[x=6.614345in,y=5.029522in,left,base]{\color{textcolor}\sffamily\fontsize{16.000000}{19.200000}\selectfont \(\displaystyle t=a/c\)}%
\end{pgfscope}%
\begin{pgfscope}%
\pgfsetroundcap%
\pgfsetroundjoin%
\pgfsetlinewidth{1.505625pt}%
\definecolor{currentstroke}{rgb}{0.172549,0.627451,0.172549}%
\pgfsetstrokecolor{currentstroke}%
\pgfsetdash{}{0pt}%
\pgfpathmoveto{\pgfqpoint{5.992123in}{4.756606in}}%
\pgfpathlineto{\pgfqpoint{6.214345in}{4.756606in}}%
\pgfpathlineto{\pgfqpoint{6.436567in}{4.756606in}}%
\pgfusepath{stroke}%
\end{pgfscope}%
\begin{pgfscope}%
\definecolor{textcolor}{rgb}{0.150000,0.150000,0.150000}%
\pgfsetstrokecolor{textcolor}%
\pgfsetfillcolor{textcolor}%
\pgftext[x=6.614345in,y=4.678828in,left,base]{\color{textcolor}\sffamily\fontsize{16.000000}{19.200000}\selectfont \(\displaystyle t=3a/2c\)}%
\end{pgfscope}%
\begin{pgfscope}%
\pgfsetroundcap%
\pgfsetroundjoin%
\pgfsetlinewidth{1.505625pt}%
\definecolor{currentstroke}{rgb}{0.839216,0.152941,0.156863}%
\pgfsetstrokecolor{currentstroke}%
\pgfsetdash{}{0pt}%
\pgfpathmoveto{\pgfqpoint{5.992123in}{4.405912in}}%
\pgfpathlineto{\pgfqpoint{6.214345in}{4.405912in}}%
\pgfpathlineto{\pgfqpoint{6.436567in}{4.405912in}}%
\pgfusepath{stroke}%
\end{pgfscope}%
\begin{pgfscope}%
\definecolor{textcolor}{rgb}{0.150000,0.150000,0.150000}%
\pgfsetstrokecolor{textcolor}%
\pgfsetfillcolor{textcolor}%
\pgftext[x=6.614345in,y=4.328134in,left,base]{\color{textcolor}\sffamily\fontsize{16.000000}{19.200000}\selectfont \(\displaystyle t=2a/c\)}%
\end{pgfscope}%
\begin{pgfscope}%
\pgfsetroundcap%
\pgfsetroundjoin%
\pgfsetlinewidth{1.505625pt}%
\definecolor{currentstroke}{rgb}{0.580392,0.403922,0.741176}%
\pgfsetstrokecolor{currentstroke}%
\pgfsetdash{}{0pt}%
\pgfpathmoveto{\pgfqpoint{5.992123in}{4.055218in}}%
\pgfpathlineto{\pgfqpoint{6.214345in}{4.055218in}}%
\pgfpathlineto{\pgfqpoint{6.436567in}{4.055218in}}%
\pgfusepath{stroke}%
\end{pgfscope}%
\begin{pgfscope}%
\definecolor{textcolor}{rgb}{0.150000,0.150000,0.150000}%
\pgfsetstrokecolor{textcolor}%
\pgfsetfillcolor{textcolor}%
\pgftext[x=6.614345in,y=3.977440in,left,base]{\color{textcolor}\sffamily\fontsize{16.000000}{19.200000}\selectfont \(\displaystyle t=5a/c\)}%
\end{pgfscope}%
\end{pgfpicture}%
\makeatother%
\endgroup%
}
\end{figure}
\end{problem}

\begin{problem}[P38 Q7]
Since $\phi$ and $\psi$ are odd functions, then
\begin{align*}
    u(x,t) &= \frac{1}{2}(\phi(ct+x) + \phi(-ct+x)) + \frac{1}{2c}\int_{-ct+x}^{ct+x} \psi(s)\d s\\
    \Rightarrow u(-x,t) &=
    \frac{1}{2}(\phi(ct-x) + \phi(-ct-x))
    +\frac{1}{2c}\int_{-ct-x}^{ct-x}\psi(s)\d s\\
    &= \frac{1}{2}(-\phi(-ct+x) - \phi(ct+x)) + 
    \frac{1}{2c}\int_{ct+x}^{-ct+x}\psi(-u)\d (-u)\\
    &= \frac{1}{2}(-\phi(-ct+x) - \phi(ct+x)) + 
    \frac{1}{2c}\int_{ct+x}^{-ct+x}\psi(u)\d u\\
    &= -\frac{1}{2}(\phi(-ct+x) + \phi(ct+x)) -
    \frac{1}{2c}\int_{-ct+x}^{ct+x}\psi(u)\d u = -u(x,t)
\end{align*}
\end{problem}

\begin{problem}[P38 Q9]
Setting 
\begin{align*}
    \xi &= x-t\\
    \eta &= 4x+t
\end{align*}
we have
\begin{align*}
    \partial_x &= \partial_\xi + 4\partial_\eta\\
    \partial_t &= -\partial_\xi + \partial_\eta
\end{align*}
\end{problem}
Then 
\begin{align*}
    \partial_{xx} - 3\partial_{xt} - 4\partial_{tt} &=
    (\partial_x - 4\partial_t)(\partial_x+\partial_t)\\
    &= 25 \partial_\xi \partial_\eta
\end{align*}
Then the general solution of the equation $\partial_\xi\partial_\eta u = 0$ would be
\begin{align*}
    u(x, t) = F(\xi) + G(\eta) = F(x-t) + G(4x+t)
\end{align*}
Then
\begin{align*}
    \phi(x) &= F(x) + G(4x)\\
    \psi(x) &= -F'(x) + G'(4x)\Rightarrow \Psi(x) = -F(x) + \frac{1}{4}G(4x)
\end{align*}
where $\Psi(x)$ is any function that $\Psi'(x)=\psi(x)$, then
\begin{align*}
    F(x) &= \frac{1}{5}(\phi(x) - 4\Psi(x))\\
    G(4x) &= \frac{4}{5}(\phi(x) + \Psi(x)) \Rightarrow G(x) = \frac{4}{5}(\phi(x/4) + \Psi(x/4)) \\
    u(x) &= \frac{1}{5}(\phi(x-t) - 4\Psi(x-t)) + \frac{4}{5}(\phi(x+t/4) + \Psi(x+t/4))\\
    &= \frac{1}{5}(\phi(x-t)+4\phi(x+t/4)) + \frac{4}{5}\int_{x-t}^{x+t/4}\psi(s)\d s
\end{align*}
According to the boundary condition $u(x, 0)=x^2$, $u_t(x, 0)=e^x$, we have
\begin{align*}
    u(x) &= \frac{1}{5}[(x-t)^2+4(x+t/4)^2] + \frac{4}{5}(e^{x+t/4} - e^{x-t})
\end{align*}

\begin{problem}[P41 Q1]
    Using the conservation law, we know that
    \begin{align*}
        E(t) = E(0) = \frac{1}{2}\int_\Omega \psi^2(x) + c^2{\phi'}^2(x) 
        = \frac{1}{2}\int_\Omega 0\d x = 0
    \end{align*}
    Then
    \begin{align*}
        E(t) = \frac{1}{2}\int_\Omega u_t^2(x, t) + c^2u_x^2(x,t) \d x = 0
        \Rightarrow u_t = 0, u_x = 0
    \end{align*}
    by the first vanish theorem.
    Then we can solve that $u=k$ for some constant $k$.
    Since $\phi(x)=u(x,0)=0$, we have $u(x,t)=u(x,0)=0$.
\end{problem}





% \end{multicols*}
\end{document}

