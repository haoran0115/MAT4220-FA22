\documentclass[twoside,11pt]{article}
\usepackage[left=1in, right=1in, top=1in, bottom=1in]{geometry}
\usepackage{amsmath}
\usepackage{amssymb}
\usepackage{amsfonts}
\usepackage{mathtools}
\usepackage{amsthm}
\usepackage{fancyhdr}
\usepackage{enumitem}
\usepackage{siunitx}
\usepackage{booktabs}
\usepackage[hidelinks]{hyperref}
\usepackage{sectsty}
\usepackage{mathrsfs} % mathscr
\usepackage{tikz}
\usepackage{pgfplots}
\usepackage{multicol}
\usepackage{listings}
\usepackage{soul}
\usepackage{braket}
\usepackage{esint}
% \usepackage{amsart}

% lmodern
\usepackage{lmodern}

% change mathcal shape
\usepackage[mathcal]{eucal}

% allow H option of figure
\usepackage{float}

% define math operators
\newcommand{\F}{\mathbb{F}}
\newcommand{\R}{\mathbb{R}}
\newcommand{\N}{\mathbb{N}}
\newcommand{\Z}{\mathbb{Z}}
\newcommand{\Q}{\mathbb{Q}}
\newcommand{\X}{\mathbb{Y}}
\renewcommand{\L}{\mathcal{L}}
% \renewcommand{\d}{\mathrm{d}}
\renewcommand*\d{\mathop{}\!\mathrm{d}}
\DeclareMathOperator*{\argmax}{arg\,max}
\DeclareMathOperator*{\argmin}{arg\,min}
\DeclareMathOperator{\im}{im}
\DeclareMathOperator{\id}{id}
\renewcommand{\mod}[1]{\ (\mathrm{mod}\ #1)}

% section font style
\sectionfont{\sffamily\Large}
\subsectionfont{\sffamily\normalsize}
\subsubsectionfont{\bf}

% line spreading and break
\hyphenpenalty=5000
\tolerance=20
\setlength{\parindent}{0em}
\setlength\parskip{0.5em}
\allowdisplaybreaks
\linespread{0.9}

% enumerate settings
% no break before enumerate
\setlist[enumerate]{itemsep=2pt,topsep=2pt}

% theorem
% latex theorem
% definition style
\theoremstyle{definition}
\newtheorem{theorem}{Theorem}[subsection]
\newtheorem{axiom}{Axiom}[section]
\newtheorem{definition}{Definition}[section]
\newtheorem{example}{Example}[section]
\newtheorem{question}{Question}[section]
\newtheorem{exercise}{Exercise}[section]
\newtheorem*{exercise*}{Exercise}
\newtheorem{lemma}{Lemma}[section]
\newtheorem{proposition}{Proposition}[section]
\newtheorem{corollary}{Corollary}[section]
\newtheorem*{theorem*}{Theorem}
\newtheorem{problem}{Problem}
% remark style
\theoremstyle{remark}
\newtheorem*{remark}{Remark}
\newtheorem*{solution}{Solution}
\newtheorem*{claim}{Claim}


% paragraph indent
\setlength{\parindent}{0em}
\setlength\parskip{0.5em}

\newcommand\Code{MAT4220 FA22}
\newcommand\Ass{HW06}
\newcommand\name{Haoran Sun}
\newcommand\mail{haoransun@link.cuhk.edu.cn}

\title{{\sffamily \Code \ \Ass}}
\author{\sffamily \name \ (\href{mailto:\mail}{\mail})}
\date{\sffamily \today}

\makeatletter
% \let\Title\@title
\let\theauthor\@author
\let\thedate\@date

\fancypagestyle{plain}{%
    \fancyhf{}
    \lhead{\sffamily \Ass}
    \rhead{\sffamily \name}
    \rfoot{\sffamily\thepage}

    % # 页脚自定义
    \fancyfoot[L]{
        \begin{minipage}[c]{0.06\textwidth}
            \includegraphics[height=7.5mm]{logo2.png}
        \end{minipage}
    }
}
\fancypagestyle{title}{%
    \fancyhf{}
    \renewcommand{\headrulewidth}{0pt}
    % \lhead{\Title}
    % \rhead{\theauthor}
    \rfoot{\sffamily\thepage}

    % # 页脚自定义
    \fancyfoot[L]{
        \begin{minipage}[c]{0.06\textwidth}
            \includegraphics[height=7.5mm]{logo2.png}
        \end{minipage}
    }
}
\fancyfootoffset[L]{0.3cm}

% re-define title format
\makeatletter
\renewcommand{\maketitle}{\bgroup\setlength{\parindent}{0pt}
\begin{flushleft}
  \textbf{\Large\@title}

  \@author
\end{flushleft}\egroup
}
\makeatother

\pagestyle{plain}

% lstlisting settings
\lstset{
    basicstyle=\linespread{0.7}\footnotesize,
    breaklines=true,
    basewidth=0.5em
}


\begin{document}
\maketitle
\thispagestyle{title}
% \begin{multicols*}{2}

% \begin{remark}
%     $V_\epsilon(x)$ is used to denote a $\epsilon$-neighborhood
%     \begin{align*}
%         V_\epsilon(x) = B_\epsilon(x)\setminus\{x\}
%     \end{align*}
% \end{remark}


\begin{problem}[P135 Q15]\
\begin{enumerate}[label=(\alph*)]
\item Note that $\|\cos(n+1/2)x\|^2=\pi/2$ on $(0, \pi)$, 
then we have
\begin{align*}
    B_n &= \frac{\pi}{2}\int_0^\pi\cos[(n+1/2)x]\d x = \frac{4}{(2n+1)\pi}(-1)^n
\end{align*}

\item The series converges for all $x$ in $(-2\pi, 2\pi)$.
\begin{align*}
    S(x) &= \begin{cases}
        1 & x\in(-\pi, \pi)\\
        -1 & x\in(-2\pi, -\pi)\cup (\pi, 2\pi)\\
        0 & x=\pm\pi
    \end{cases}
\end{align*}

\item Using Parseval's equality, we have
\begin{align*}
    \sum_{n=0} Bn^2\|X_n(x)\|^2 &= \|\phi(x)\|^2\\
    \Rightarrow \sum_{n=0} \frac{16}{\pi}\frac{1}{(2n+1)^2}\frac{\pi}{2} &= \pi\\
    \sum_{n=0} \frac{1}{(2n+1)^2} &= \frac{\pi}{8}
\end{align*}

\end{enumerate}
\end{problem}


\begin{problem}[P145 Q4]\
\begin{enumerate}[label=(\alph*)]
\item Let $T(t)X(t)$ satisfies the boundary condition, then
\begin{align*}
    \frac{T'(t)}{kT} &= \frac{X''(x)}{X(x)} = -\lambda
\end{align*}
Assume that $\lambda\geq 0$.
If $\lambda=0$, we have $X_0(x) = A + Bx$.
Else, let $\beta^2=\lambda> 0$, then we have the following form of 
the eigenfunctions.
\begin{align*}
    X(x) &= C\cos\beta x + D\sin\beta x
\end{align*}
Applying the boundary condition, we can solve that
\begin{align*}
    \tan\frac{\beta_n l}{2} &= \frac{\beta_n l}{2}
\end{align*}
Hence $u(x, t)$ could be written in the form of
\begin{align*}
    u(x, t) &= 
    A + Bx + 
    \sum_{n=1}^\infty e^{-\beta_n^2kt}(c_n\cos\beta_nx + d_n\sin\beta_n x)
\end{align*}

\item Suppose we can take the limit term by term, hence
\begin{align*}
    \lim_{t\rightarrow \infty}
    \sum_{n=1}^\infty e^{-\beta_n^2kt}(c_n\cos\beta_nx + d_n\sin\beta_n x)
    &= \sum_{n=1}^\infty
    \lim_{t\rightarrow \infty}
    e^{-\beta_n^2kt}(c_n\cos\beta_nx + d_n\sin\beta_n x)\\
    &= \sum_{n=1}^\infty 0 = 0
\end{align*}
Consequently
\begin{align*}
    \lim_{t\rightarrow \infty} u(x,t) = A + Bx
\end{align*}

\item Suppose $\lambda=-\beta^2<0$.
Note that
\begin{align*}
    \int_0^l u_x(x, t)\d x &= \left .u(x,t)u_x(x, t)\right |_{x=0}^{x=l}
    - \int_0^l u(x, t)u_{xx}(x, t)\d x \geq 0
    \Rightarrow \int_0^l u(x, t)u_{xx}(x, t)\d x \leq 0
\end{align*}
However
\begin{align*}
    \int_0^l u(x, t)u_{xx}(x, t)\d x &= T(t)^2\int_0^l X(x)X''(x)\d x
    = T(t)^2\beta^2 \int_0^l X^2(x)\d x \geq 0 
\end{align*}
where the contradiction occurs that $\lambda = 0$ in this case.
Hence $\lambda>0$.

\item Since $\braket{1,1}=l$, $\braket{x,x}=l^3/3$, then
\begin{align*}
    A &= \frac{1}{l}\int_0^l\phi(x)\d x \quad
    B = \frac{3}{l^3}\int_0^l x\phi(x)\d x
\end{align*}

\end{enumerate}
\end{problem}


\begin{problem}[P145 Q6]
Suppose $u$ is in the form of 
\begin{align*}
    u(x,t) &= \frac{1}{2}A_0 + 
    \sum_{n=1}^\infty A_ne^{-\frac{n^2\pi^2}{l^2}kt}
    \cos\frac{n\pi x}{l}
\end{align*}
where $u(x,0)=\phi(x)$ continuous on $[0, l]$.
\begin{claim}
$A_n$ bounded.
\end{claim}
\begin{proof}
Since $\phi$ is continuous on $[0, l]$, then $\|\phi\|$ bounded, hence
\begin{align*}
    |\braket{\phi, \cos\frac{n\pi}{l}x}| &\leq
    \|\phi\| \|\cos\frac{n\pi}{l}\| < \infty
    \Rightarrow
    A_n = \frac{2}{l}\braket{\cos\frac{n\pi}{l},\phi} < M\qedhere
\end{align*}
\end{proof}

\begin{claim}
The following series converges $\forall t>0$
\begin{align*}
    \sum_{n=1}^\infty A_n n^k e^{-n^2t}
\end{align*}
\end{claim}
\begin{proof}
Note that $\forall\epsilon>0$, $\exists N\in\N$ s.t.
$\forall m > N$ we have
$\sum_N^\infty n^k e^{-nt} < \epsilon/M$, then
\begin{align*}
    \sum_{n=1}^\infty |A_n n^k e^{-n^2t}|
    \leq M \sum_{n=1}^\infty n^k e^{-n^2 t}
    < M\sum_{n=1}^{m-1} n^k e^{-nt} + \epsilon
\end{align*}
Hence the series converges.
\end{proof}
According to the claims, $\exists N\in\N$ s.t. 
$\forall x $ and $\forall m>N$ we have
\begin{align*}
    \left|\sum_{m}^\infty
    \frac{\d^k}{\d x^k}A_n e^{-\frac{n^2\pi^2}{l^2}kt}\cos\frac{n\pi x}{l}
    \right| 
    \leq 
    M\sum_{n=m}^\infty e^{-\frac{n^2\pi^2}{l^2}kt}
    < \epsilon
\end{align*}
which means the series converges uniformly with respect to $x$.
Hence, using the theorem in the appendix,
we have
\begin{align*}
    \frac{\d^k}{\d x^k}u(x, t) &= 
    \frac{\d^k}{\d x^k}
    \sum_{n=1}^\infty
    A_n e^{-\frac{n^2\pi^2}{l^2}kt}\cos\frac{n\pi x}{l}
    = 
    \sum_{n=1}^\infty
    \frac{\d^k}{\d x^k}
    A_n e^{-\frac{n^2\pi^2}{l^2}kt}\cos\frac{n\pi x}{l}\\
    &= 
    \sum_{n=1}^\infty
    B_n n^k e^{-\frac{n^2\pi^2}{l^2}kt}\cos\frac{n\pi x}{l} 
\end{align*}
exists $\forall k$ in $t>0$.

\end{problem}


\begin{problem}[P145 Q11]
Follow the same steps when proving the uniform convergence.
Since the $f'(x)$ piecewise continuous $f'(x)X(x)$ also piecewise continuous,
then it is integrable and hence
\begin{align*}
    A_n &= -\frac{1}{n}B_n'\quad B_n = \frac{1}{n}A_n'
\end{align*}
Applying Bessel's inequality, we have the following series which is convergent
\begin{align*}
    \sum_{n=1}^\infty ({A_n'}^2 + {B_n'}^2) < \infty
\end{align*}
Therefore 
\begin{align*}
    \left|\sum_{n=1}^\infty A_n\cos nx + B_n\sin nx\right|
    &\leq \sum_{n=1}^\infty |A_n\cos nx| + |B_n\sin nx|\\
    &\leq \sum_{n-1}^\infty |A_n| + |B_n|\\
    &= \frac{1}{n}\sum_{n-1}^\infty |A_n'| + |B_n'|\\
    &\leq \left(\sum_{n=1}^\infty \frac{1}{n^2}\right)^{1/2}
    \left(\sum_{n=1}^\infty |A_n'|^2 + |B_n'|^2 + 2|A_n'||B_n'|\right)^{1/2}\\
    &\leq \left(\sum_{n=1}^\infty \frac{1}{n^2}\right)^{1/2}
    \left(\sum_{n=1}^\infty 2(|A_n'|^2 + |B_n'|^2)\right)^{1/2}<\infty
\end{align*}
Hence $\forall \epsilon>0$, we can choose $N\in\N$ s.t. $\forall m > N$
and $\forall x$ we have
\begin{align*}
    \left|f(x) - \sum_{n=1}^{m-1} A_n\cos nx + B_n\sin nx\right|
    &= \left|\sum_{n=m}^\infty A_n\cos nx + B_n\sin nx\right|\\
    &\leq M\sum_{n=m}^\infty (|A_n'|^2 + |B_n'|^2) < \epsilon
\end{align*}
Hence the Fourier series converges uniformly.
\end{problem}



\begin{problem}[P160 Q5]
Note that $(x^2+y^2)/4+c$ is the solution of $\Delta u = 1$.
Since $(x^2+y^2)/4 - 1/4$ satisfies the boundary condition, 
according to the uniqueness, the solution is $u(x, t)=(x^2+y^2)/4 - 1/4$.
\end{problem}



\begin{problem}[P160 Q11]
Suppose there is a solution $u$, then
\begin{align*}
    \iiint_D f\d x\d y\d z &= 
    \iiint_D\nabla\cdot\nabla u\d x\d y\d z
    = \oiint_{\partial D}\nabla u\cdot\mathbf{n}\d \sigma
    = \oiint_{\partial D} \frac{\partial u}{\partial \mathbf{n}}\d \sigma
    = \oiint_{\partial D} g\d \sigma
\end{align*}
Then if equality does not hold, there will be no solutions.
\end{problem}


\begin{problem}[P160 Q13]
Let $v=u+\epsilon|\mathbf{x}|^2$.
Suppose $v$ obtains its maximum in the interior domain of $D$, then
$\Delta v \leq 0$.
Note that
\begin{align*}
    \Delta v &= \Delta u + 4n\epsilon > 0
\end{align*}
where $n$ is the dimension.
This contradicts the assumption.
Hence $\max_D v = \max_{\partial D} v$.
Since $D$ is bounded, we can also show that
\begin{align*}
    \max_D v = \max_D u\quad
    \max_{\partial D} v = \max_{\partial D} u
\end{align*}
Hence $\max_{D} u = \max_{\partial D} u$.


\end{problem}




% \end{multicols*}
\end{document}

