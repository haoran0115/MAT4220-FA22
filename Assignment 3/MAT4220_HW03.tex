\documentclass[twoside,11pt]{article}
\usepackage[left=1in, right=1in, top=1in, bottom=1in]{geometry}
\usepackage{amsmath}
\usepackage{amssymb}
\usepackage{amsfonts}
\usepackage{mathtools}
\usepackage{amsthm}
\usepackage{fancyhdr}
\usepackage{enumitem}
\usepackage{siunitx}
\usepackage{booktabs}
\usepackage[hidelinks]{hyperref}
\usepackage{sectsty}
\usepackage{mathrsfs} % mathscr
\usepackage{tikz}
\usepackage{pgfplots}
\usepackage{multicol}
\usepackage{listings}
% \usepackage{amsart}

% change mathcal shape
\usepackage[mathcal]{eucal}

% allow H option of figure
\usepackage{float}

% define math operators
\newcommand{\F}{\mathbb{F}}
\newcommand{\R}{\mathbb{R}}
\newcommand{\N}{\mathbb{N}}
\newcommand{\Z}{\mathbb{Z}}
\newcommand{\Q}{\mathbb{Q}}
\newcommand{\X}{\mathbb{Y}}
\renewcommand{\L}{\mathcal{L}}
% \renewcommand{\d}{\mathrm{d}}
\renewcommand*\d{\mathop{}\!\mathrm{d}}
\DeclareMathOperator*{\argmax}{arg\,max}
\DeclareMathOperator*{\argmin}{arg\,min}
\DeclareMathOperator{\im}{im}
\DeclareMathOperator{\id}{id}
\renewcommand{\mod}[1]{\ (\mathrm{mod}\ #1)}

% section font style
\sectionfont{\sffamily\Large}
\subsectionfont{\sffamily\normalsize}
\subsubsectionfont{\bf}

% line spreading and break
\hyphenpenalty=5000
\tolerance=20
\setlength{\parindent}{0em}
\setlength\parskip{0.5em}
\allowdisplaybreaks
\linespread{0.9}

% theorem
% latex theorem
% definition style
\theoremstyle{definition}
\newtheorem{theorem}{Theorem}[subsection]
\newtheorem{axiom}{Axiom}[section]
\newtheorem{definition}{Definition}[section]
\newtheorem{example}{Example}[section]
\newtheorem{question}{Question}[section]
\newtheorem{exercise}{Exercise}[section]
\newtheorem*{exercise*}{Exercise}
\newtheorem{lemma}{Lemma}[section]
\newtheorem{proposition}{Proposition}[section]
\newtheorem{corollary}{Corollary}[section]
\newtheorem*{theorem*}{Theorem}
\newtheorem{problem}{Problem}
% remark style
\theoremstyle{remark}
\newtheorem*{remark}{Remark}
\newtheorem*{solution}{Solution}
\newtheorem*{claim}{Claim}


% paragraph indent
\setlength{\parindent}{0em}
\setlength\parskip{0.5em}

\newcommand\Code{MAT4220 FA22}
\newcommand\Ass{HW03}
\newcommand\name{Haoran Sun}
\newcommand\mail{haoransun@link.cuhk.edu.cn}

\title{{\sffamily \Code \ \Ass}}
\author{\sffamily \name \ (\href{mailto:\mail}{\mail})}
\date{\sffamily \today}

\makeatletter
% \let\Title\@title
\let\theauthor\@author
\let\thedate\@date

\fancypagestyle{plain}{%
    \fancyhf{}
    \lhead{\sffamily \Ass}
    \rhead{\sffamily \name}
    \rfoot{\sffamily\thepage}

    % # 页脚自定义
    \fancyfoot[L]{
        \begin{minipage}[c]{0.06\textwidth}
            \includegraphics[height=7.5mm]{logo2.png}
        \end{minipage}
    }
}
\fancypagestyle{title}{%
    \fancyhf{}
    \renewcommand{\headrulewidth}{0pt}
    % \lhead{\Title}
    % \rhead{\theauthor}
    \rfoot{\sffamily\thepage}

    % # 页脚自定义
    \fancyfoot[L]{
        \begin{minipage}[c]{0.06\textwidth}
            \includegraphics[height=7.5mm]{logo2.png}
        \end{minipage}
    }
}
\fancyfootoffset[L]{0.3cm}

% re-define title format
\makeatletter
\renewcommand{\maketitle}{\bgroup\setlength{\parindent}{0pt}
\begin{flushleft}
  \textbf{\Large\@title}

  \@author
\end{flushleft}\egroup
}
\makeatother

\pagestyle{plain}

% lstlisting settings
\lstset{
    basicstyle=\linespread{0.7}\footnotesize,
    breaklines=true,
    basewidth=0.5em
}


\begin{document}
\maketitle
\thispagestyle{title}
% \begin{multicols*}{2}

% \begin{remark}
%     $V_\epsilon(x)$ is used to denote a $\epsilon$-neighborhood
%     \begin{align*}
%         V_\epsilon(x) = B_\epsilon(x)\setminus\{x\}
%     \end{align*}
% \end{remark}

\begin{problem}[P52 Q3]
Using the formula
\begin{align*}
    u(x,t) &= \Phi\star\phi(x) = \frac{1}{\sqrt{4\pi kt}}\int_\R
    e^{-\frac{(x-y)^2}{4kt}} e^{3y}\d y\\
    &= \frac{1}{\sqrt{4\pi kt}}\int_\R e^{-\frac{(y-x-6kt)^2}{4kt}}
    e^{\frac{6kt(x+6kt)}{4kt}}\d y\\
    &= e^{3x+9kt}
\end{align*}
\end{problem}


\begin{problem}[P53 Q14]
Suppose $1\neq 4kta$, we have
\begin{align}
    |u(x,t)| &= |\Phi\star\phi(x)|\leq |\Phi\star Ce^{ax^2}|\\
    &= \frac{1}{\sqrt{4\pi kt}}\int_\R
    \exp\left[
        -\frac{(1-4kta)(y-\frac{x}{1-4kta})^2}{4kt}
    \right]
    \exp\frac{ax^2}{4kta - 1}\d x
    \label{eq:q2}
\end{align}
For $t\in(0,1/(4ak))$, we have $1-4kat>0$, then
\begin{align*}
    |u(x,t)| \leq \frac{1}{\sqrt{1-4kat}}\exp\frac{ax^2}{4kta - 1}\d x
\end{align*}
This upper bound makes sense $\forall x\in\Omega$, which means
$u(x,t)$ is meaningful.

However, when $t\geq 1/(4kat)$, $u(x,t)$ may not make any sense.
For example, let $\phi(x)=Ce^{ax^2}\leq Ce^{ax^2}$, according to equation \ref{eq:q2},
we can see that $u(x,t)=\infty$ when $t>1/(4ak)$,
a $e^{ky^2}$ ($k>0$) term appears in the integral term, which means
the integral not exists (goes to infinity).
\end{problem}


\begin{problem}[P60 Q3]
Define
\begin{align*}
    \tilde{\phi}(x) &= \phi(|x|)
\end{align*}
Then 
\begin{align*}
    \tilde{w}(x,t) &= \frac{1}{\sqrt{4\pi kt}}\int_\R
    \tilde{\phi}(y)e^{-\frac{(x-y)^2}{4kt}}
    =\frac{1}{\sqrt{4\pi kt}}\int_0^\infty\left[
        e^{-\frac{(x+y)^2}{4kt}}
        +e^{-\frac{(x-y)^2}{4kt}}
    \right]\phi(y) \d y
\end{align*}
\end{problem}


\begin{problem}[P66 Q3]
Perform the odd extension to both
$\phi(x)$ and $\psi(x)$ to $\tilde{\phi}(x)$
and $\tilde{\psi}(x)$, then we get
\begin{align*}
    u(x,t) = \frac{1}{2}[
        \tilde{\phi}(x-ct)
        + \tilde{\phi}(x+ct)
    ] + \frac{1}{2c}
    \int_{x-ct}^{x+ct}\tilde{\psi}(s)\d s
\end{align*}
Plugin $\phi(x)=f(x)$, $\psi(x)=cf'(x)$,
then $\forall x\geq 0$, we have two cases
\begin{enumerate}
    \item $x> ct\geq 0$, we get
    \begin{align*}
        u(x,t) &= \frac{1}{2}[
            f(x-ct) + f(x+ct)
        ] + \frac{1}{2c}
        \int_{x-ct}^{x+ct}cf'(s)\d s = 
        f(x+ct)
    \end{align*}

    \item $0\leq x\leq ct$, we get
    \begin{align*}
        u(x,t) 
        &= \frac{1}{2}[
            -f(ct-x) + f(x+ct)
        ] + \frac{1}{2c}\left[
            \int_{x-ct}^0-cf'(-s)\d s
            + \int_0^{x+ct}cf'(s)\d s
        \right]\\
        &= \frac{1}{2c}[
            -f(ct-x) + f(x+ct)
        ] + \frac{1}{2}\left[
            -cf(ct-x) + cf(x+ct)
        \right]\\
        &= f(x+ct) - f(ct-x)
    \end{align*}
\end{enumerate}
\end{problem}


\begin{problem}[P67 Q10]
Perform odd extension of $\phi$ and $\psi$ on $(0, \pi)$
and perform even extension on whole $\R$, we get
\begin{align*}
    \tilde{\phi}(x) &= \cos x,\ \tilde{\psi}(x) = 0
\end{align*}
Then the solution of this wave equation is
\begin{align*}
    u(x,t) &=
    \frac{1}{2}(\cos (x-ct) + \cos (x+ct)) 
    = \cos x\cos ct = \cos x\cos 3t
\end{align*}
using the fact that $c=3$.
\end{problem}


\begin{problem}[P79 Q2]
Using the formula that
\begin{align*}
    u(x,t) &= \frac{1}{2}[\phi(x-ct)+\phi(x+ct)]
    +\frac{1}{2c}\int_{x-ct}^{x-ct}\psi(s)\d s
    +\int_0^t\int_{x-c(t-s)}^{x+c(t-s)} f(y, s)\d y \d s
\end{align*}
we have
\begin{align*}
    u(x, t)&= \frac{1}{2c}\int_0^t\int_{x-c(t-s)}^{x+c(t-s)} e^{ay}\d y\d s\\
    &= \frac{1}{2ac}\int_0^t(
        e^{ax+ac(t-s)} - e^{ax-ac(t-s)}
    )\d s\\
    &= \frac{1}{2a^2c^2}(e^{ax+act}+e^{ax-act} - 2e^{ax})\\
    &= \frac{1}{2a^2c^2}e^{ax}(e^{act} + e^{-act} - 2)
\end{align*}
\end{problem}



\begin{problem}[P79 Q14]
Define a new function $v(x,t) = u(x,t) - xk(t)$, then we have
\begin{align*}
    v_{tt} - c^2v_{xx} &= f(x, t) = - xk''(t)\\
    v(x, 0) &= \phi(x) = -xk(0)\\
    v_t(x,0) &= \psi(x) = -xk'(0)\\
    v_x(0, t) &= 0\text{\ N.B.C}
\end{align*}
on $x\in (0, \infty)$.
Then we can perform even extension on $\phi$, $\psi$ and $f$.
Applying the formula
\begin{align*}
    \tilde{u}(x, t) &= {\color{red} \frac{1}{2}[\tilde{\phi}(x+ct) + \tilde{\phi}(x-ct)]}
    + {\color{blue} \frac{1}{2c}\int_{x-ct}^{x+ct} \psi(s)\d s}
    + {\color{orange} \frac{1}{2c}\int_0^t\d s \int_{x-c(t-s)}^{x+c(t-s)}\d y f(y, s)}
\end{align*}
When $0\leq ct\leq x$, the three parts red $r$, blud $b$, and orange $o$ equals to
\begin{align*}
    r &= -xk(0)\\
    b &= -xtk'(0)\\
    o &= -xk(t) + xk(0) + txk'(0)
\end{align*}
Then $v(x, t) = -xk(t)$, which means $u(x, t) = 0$ when $0\leq ct\leq x$.

When $ct>x\geq 0$, the red part $r$ equals to
\begin{align*}
    r = \frac{1}{2}[-(x+ct)k(0) - (ct-x)k(0)] = -ctk(0)
\end{align*}
The blue part $b$ equals to
\begin{align*}
    b &= \frac{1}{2c}\int_0^{x+ct}-k'(0)s \d s + \frac{1}{2c}\int_{x-ct}^0 k'(0)s \d s = -\frac{1}{2c}
    (x^2+c^2t^2)k'(0)
\end{align*}
The orange part $o$ equals to
\begin{align*}
    o &= \left(
        \int_0^{t-x/c} + \int_{t-x/c}^t
    \right)\d s \int_{x-c(t-s)}^{x+c(t-s)}\d y f(y, s)\\
    &= -\frac{1}{2c}\int_0^{t-x/c}\d s [x^2+c^2(t-s)^2]k''(s) + \int_{t-x/c}^t x(s-t)k''(s)\d s\\
    &= -\frac{1}{2c}x^2[k'(t-x/c) - k'(0)] - \frac{c}{2}(s-t)^2k'(s)|_0^{t-x/c}
    + c(s-t)k(s)|_0^{t-x/c} - c\int_0^{t-x/c}k(s)\d s \\
    &+\ x(s-t)k'(s)|_{t-x/c}^t - xk(s)|_{t-x/c}^t\\
    &= \frac{1}{c}x^2k'(t-x/c) - xk(t) + xk(t-x/c) - \frac{1}{2c}x^2k'(t-x/c) + \frac{1}{2c}x^2k'(0)\\
    &-\ \frac{1}{2c}x^2k'(t-x/c) + \frac{c}{2}t^2k'(0) - xk(t-x/c) + ctk(0) 
    - c\int_0^{t-x/c}k(s)\d s\\
    &= -xk(t) + \frac{1}{2c}x^2k'(0) + \frac{c}{2}t^2k'(0) + ctk(0) - c\int_0^{t-x/c}k(s)\d s
\end{align*}
Therefore we have
\begin{align*}
    v(x, t)&= r + b + o \\
    & = -ctk(0)
    -\frac{1}{2c}(x^2+c^2t^2)k'(0)
    -xk(t) + \frac{1}{2c}x^2k'(0) + \frac{c}{2}t^2k'(0) + ctk(0) - c\int_0^{t-x/c}k(s)\d s\\
    &= -xk(t) - c\int_0^{t-x/c}k(s)\d s\\
    \Rightarrow u(x, t) &= v(x, t) + xk(t) = -c\int_{0}^{t-x/c} k(s)\d s
\end{align*}
for $ct>x>0$.


\end{problem}


% \end{multicols*}
\end{document}

