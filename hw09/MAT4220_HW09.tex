\documentclass[twoside,11pt]{article}
\usepackage[left=1in, right=1in, top=1in, bottom=1in]{geometry}
\usepackage{amsmath}
\usepackage{amssymb}
\usepackage{amsfonts}
\usepackage{mathtools}
\usepackage{amsthm}
\usepackage{fancyhdr}
\usepackage{enumitem}
\usepackage{siunitx}
\usepackage{booktabs}
\usepackage[hidelinks]{hyperref}
\usepackage{sectsty}
\usepackage{mathrsfs} % mathscr
\usepackage{tikz}
\usepackage{pgfplots}
\usepackage{multicol}
\usepackage{listings}
\usepackage{soul}
\usepackage{braket}
\usepackage{esint}
% \usepackage{amsart}

% lmodern
\usepackage{lmodern}

% change mathcal shape
\usepackage[mathcal]{eucal}

% allow H option of figure
\usepackage{float}

% define math operators
\newcommand{\F}{\mathbb{F}}
\newcommand{\R}{\mathbb{R}}
\newcommand{\N}{\mathbb{N}}
\newcommand{\Z}{\mathbb{Z}}
\newcommand{\Q}{\mathbb{Q}}
\newcommand{\X}{\mathbb{Y}}
\renewcommand{\L}{\mathcal{L}}
% \renewcommand{\d}{\mathrm{d}}
\renewcommand*\d{\mathop{}\!\mathrm{d}}
\DeclareMathOperator*{\argmax}{arg\,max}
\DeclareMathOperator*{\argmin}{arg\,min}
\DeclareMathOperator{\im}{im}
\DeclareMathOperator{\id}{id}
\renewcommand{\mod}[1]{\ (\mathrm{mod}\ #1)}

% section font style
\sectionfont{\sffamily\Large}
\subsectionfont{\sffamily\normalsize}
\subsubsectionfont{\bf}

% line spreading and break
\hyphenpenalty=5000
\tolerance=20
\setlength{\parindent}{0em}
\setlength\parskip{0.5em}
\allowdisplaybreaks
\linespread{0.9}

% enumerate settings
% no break before enumerate
\setlist[enumerate]{itemsep=2pt,topsep=2pt}

% theorem
% latex theorem
% definition style
\theoremstyle{definition}
\newtheorem{theorem}{Theorem}[subsection]
\newtheorem{axiom}{Axiom}[section]
\newtheorem{definition}{Definition}[section]
\newtheorem{example}{Example}[section]
\newtheorem{question}{Question}[section]
\newtheorem{exercise}{Exercise}[section]
\newtheorem*{exercise*}{Exercise}
\newtheorem{lemma}{Lemma}[section]
\newtheorem{proposition}{Proposition}[section]
\newtheorem{corollary}{Corollary}[section]
\newtheorem*{theorem*}{Theorem}
\newtheorem{problem}{Problem}
% remark style
\theoremstyle{remark}
\newtheorem*{remark}{Remark}
\newtheorem*{solution}{Solution}
\newtheorem*{claim}{Claim}


% paragraph indent
\setlength{\parindent}{0em}
\setlength\parskip{0.5em}

\newcommand\Code{MAT4220 FA22}
\newcommand\Ass{HW09}
\newcommand\name{Haoran Sun}
\newcommand\mail{haoransun@link.cuhk.edu.cn}

\title{{\sffamily \Code \ \Ass}}
\author{\sffamily \name \ (\href{mailto:\mail}{\mail})}
\date{\sffamily \today}

\makeatletter
% \let\Title\@title
\let\theauthor\@author
\let\thedate\@date

\fancypagestyle{plain}{%
    \fancyhf{}
    \lhead{\sffamily \Ass}
    \rhead{\sffamily \name}
    \rfoot{\sffamily\thepage}

    % # 页脚自定义
    \fancyfoot[L]{
        \begin{minipage}[c]{0.06\textwidth}
            \includegraphics[height=7.5mm]{logo2.png}
        \end{minipage}
    }
}
\fancypagestyle{title}{%
    \fancyhf{}
    \renewcommand{\headrulewidth}{0pt}
    % \lhead{\Title}
    % \rhead{\theauthor}
    \rfoot{\sffamily\thepage}

    % # 页脚自定义
    \fancyfoot[L]{
        \begin{minipage}[c]{0.06\textwidth}
            \includegraphics[height=7.5mm]{logo2.png}
        \end{minipage}
    }
}
\fancyfootoffset[L]{0.3cm}

% re-define title format
\makeatletter
\renewcommand{\maketitle}{\bgroup\setlength{\parindent}{0pt}
\begin{flushleft}
  \textbf{\Large\@title}

  \@author
\end{flushleft}\egroup
}
\makeatother

\pagestyle{plain}

% lstlisting settings
\lstset{
    basicstyle=\linespread{0.7}\footnotesize,
    breaklines=true,
    basewidth=0.5em
}


\begin{document}
\maketitle
\thispagestyle{title}
% \begin{multicols*}{2}

% \begin{remark}
%     $V_\epsilon(x)$ is used to denote a $\epsilon$-neighborhood
%     \begin{align*}
%         V_\epsilon(x) = B_\epsilon(x)\setminus\{x\}
%     \end{align*}
% \end{remark}

\begin{problem}[P197 Q11]\
\begin{enumerate}[label=(\alph*)]
\item Easy to verify that (18) satisfies $\Delta G = 0$ except at $\mathbf{x}=\mathbf{x}_0$,
$G(\mathbf{x})|_{\partial D} = 0$, $G(\mathbf{x}) - \log|\mathbf{x}-\mathbf{x}_0|/2\pi$ finite
at $\mathbf{x}_0$.

\item Note that
\begin{align*}
    \nabla G &= \frac{1}{2\pi}\frac{1}{\rho}(\mathbf{x}-\mathbf{x}_0)
    - \frac{1}{2\pi}\frac{1}{\rho^*}(\mathbf{x}-\mathbf{x}_0^*)
\end{align*}
Since $\mathbf{n} = \mathbf{x}/|\mathbf{x}|$, we have
\begin{align*}
    \nabla G\cdot\mathbf{n} &= \frac{1}{2\pi}\frac{1}{\rho}(a-r_0\cos\phi)
    - \frac{1}{2\pi}\frac{1}{\rho^*}(a-r_0^*\cos\phi)\\
    &= \frac{1}{2\pi}\frac{a-r_0a\cos\phi}{a^2+r_0^2-2ar_0\cos\phi}
    -\frac{1}{2\pi}\frac{a-\frac{a^2}{r_0}\cos\phi}{a^2+\frac{a^4}{r_0^2}-2\frac{a^3}{r_0}\cos\phi}\\
    &= \frac{1}{2\pi}\frac{1}{a}\frac{a^2-r_0a\cos\phi - r_0^2 + r_0a\cos\phi}{a^2+r_0^2-2ar_0\cos\phi}\\
    &= \frac{a^2-r_0^2}{2\pi a}\frac{1}{a^2+r_0^2-2ar_0\cos\phi}
\end{align*}
Therefore we have proved Poisson's formula since
\begin{align*}
    u(\mathbf{x}_0) &= 
    \frac{a^2-r_0^2}{2\pi a}
    \oiint_{\partial D}\d\sigma\ 
    u(\mathbf{x})\frac{\partial}{\partial \mathbf{n}}G(\mathbf{x},\mathbf{x}_0)
    =
    \frac{a^2-r_0^2}{2\pi a}
    \oiint_{\partial D}\d\sigma \
    \frac{u(\mathbf{x})}{a^2+r_0^2-2ar_0\cos\phi}
\end{align*}

\end{enumerate}
\end{problem}


\begin{problem}[P197 Q13]
The Green's function for the whole ball is
\begin{align*}
    G(\mathbf{x}, \mathbf{x}_0) &= -\frac{1}{2\pi\rho}
    + \frac{a}{|\mathbf{x}_0|}\frac{1}{4\pi\rho^*}
\end{align*}
Reflect the green's function wrt $xy$ plane, we have
\begin{align*}
    G(\mathbf{x}, \mathbf{x}_0) &= 
    -\frac{1}{4\pi\rho} + \frac{a}{|\mathbf{x}_0|}\frac{1}{4\pi\rho^*}
    +\frac{1}{4\pi\rho_z} - \frac{a}{|\mathbf{x}_0|}\frac{1}{4\pi\rho^*_z}
\end{align*}
where
\begin{align*}
    \rho &= |\mathbf{x}-\mathbf{x}_0|\quad
    \rho^* = |\mathbf{x}-\mathbf{x}_0^*|\quad
    \rho_z = |\mathbf{x}-\mathbf{x}_{0z}|\quad
    \rho_z^* = |\mathbf{x}-\mathbf{x}_{0z}^*|
\end{align*}
where $x_0^*=a^2\mathbf{x_0}/|\mathbf{x}_0|^2$,
and $\mathbf{x}_{0z}$ is the reflection of $\mathbf{x}_0$ wrt $xy$ plane,
$\mathbf{x}^*_{0z}$ is the reflection of $\mathbf{x}_0^*$ wrt $xy$ plane.

\end{problem}


\begin{problem}[P337 Q1]
Easy to prove the linearity.
To prove the continuity, since $f$ integrable on $\Omega$,
then $\forall\phi_N\rightarrow\phi$, $\phi_N\in C^\infty(\Omega)$
compactly supported, let $F=|\braket{|f|, 1}|$ on $\Omega$
(since $|f|$ also integrable), then
$\forall \epsilon>0$, $\exists N\in\N$ s.t. $\forall n>N$,
we have $|\phi_N(x)-\phi(x)|<\epsilon/F$, and hence
\begin{align*}
    |\braket{f,\phi_n}-\braket{f,\phi}|
    &=\left|
        \int_\Omega\d x\ f(x)[\phi_n(x)-\phi(x)]
    \right|\\
    &<\int_\Omega\d x\ |f(x)|\frac{\epsilon}{F}=\epsilon
\end{align*}
which means the map is continuous.

\end{problem}


\begin{problem}[P337 Q2]
\textit{Linearity:} direct prove by definition
\begin{align*}
    \braket{f',a\phi + b\psi} 
    &=
    -\braket{f,a\phi'+b\psi'}
    =-\int_\Omega\d x\ f(x)(a\phi'(x)+b\psi'(x))
    = -a\braket{f,\phi'} - b\braket{f,\psi'}\\
    &= a\braket{f',\phi} + b\braket{f',\psi}
\end{align*}

\textit{Continuity:} since $\phi_N\rightarrow\phi$ uniformly 
and $\phi_N\in C^\infty(\Omega)$ compactly supported,
then $\phi_N'\rightarrow\phi'$ uniformly and
$\phi_N'\in C^\infty(\Omega)$ compactly supported,
hence
\begin{align*}
    \braket{f,\phi_N'}\rightarrow
    \braket{f,\phi'}
    \Rightarrow
    \braket{f',\phi_N}\rightarrow
    \braket{f',\phi}
\end{align*}

\end{problem}


\begin{problem}[P337 Q5]
\begin{claim}
$-c\braket{H_x, \phi} = \braket{H_t,\phi}$.
\end{claim}
\begin{proof}
Since
\begin{align*}
    \braket{H_x,\phi} &= 
    -\iint_\Omega \d x\d t\ 
    H(x-ct)\phi_x(x, t)
    = -\int_0^\infty\d t\int_{ct}^\infty\d x\ \phi_x(x, t)
    = \int_0^\infty\d t\ \phi(ct, t)\\
    \braket{H_t,\phi} &= 
    -\iint_\Omega\d x \d t\
    H(x-ct)\phi_t(x, t)
    = -\int_0^\infty\d x\int_0^{x/c}\d t\
    \phi_t(x, t)
    = -\int_0^\infty\d x\ \phi(x, x/c)\\
    &= -c\int_0^\infty\d u\ \phi(cu, u)
\end{align*}
Hence $c\braket{H_x, \phi} = \braket{H_t,\phi}$.
\end{proof}
Therefore
\begin{align*}
    c^2\braket{H_{xx}, \phi} &= 
    -c^2\braket{H_x, \phi_x}
    = c\braket{H_t, \phi_x}
    = -c\braket{H, \phi_{xt}}\\
    \braket{H_{tt}, \phi} &= 
    -\braket{H_t, \phi_t}
    = c\braket{H_x, \phi_t}
    = -c\braket{H, \phi_{tx}}
\end{align*}
Therefore we have $\braket{H_{tt},\phi}=c^2\braket{H_{xx}, \phi}$,
which means that $H(x-ct)$ is a weak solution.

\end{problem}




% \end{multicols*}
\end{document}

