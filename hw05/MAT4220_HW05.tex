\documentclass[twoside,11pt]{article}
\usepackage[left=1in, right=1in, top=1in, bottom=1in]{geometry}
\usepackage{amsmath}
\usepackage{amssymb}
\usepackage{amsfonts}
\usepackage{mathtools}
\usepackage{amsthm}
\usepackage{fancyhdr}
\usepackage{enumitem}
\usepackage{siunitx}
\usepackage{booktabs}
\usepackage[hidelinks]{hyperref}
\usepackage{sectsty}
\usepackage{mathrsfs} % mathscr
\usepackage{tikz}
\usepackage{pgfplots}
\usepackage{multicol}
\usepackage{listings}
% \usepackage{amsart}

% lmodern
\usepackage{lmodern}

% change mathcal shape
\usepackage[mathcal]{eucal}

% allow H option of figure
\usepackage{float}

% define math operators
\newcommand{\F}{\mathbb{F}}
\newcommand{\R}{\mathbb{R}}
\newcommand{\N}{\mathbb{N}}
\newcommand{\Z}{\mathbb{Z}}
\newcommand{\Q}{\mathbb{Q}}
\newcommand{\X}{\mathbb{Y}}
\renewcommand{\L}{\mathcal{L}}
% \renewcommand{\d}{\mathrm{d}}
\renewcommand*\d{\mathop{}\!\mathrm{d}}
\DeclareMathOperator*{\argmax}{arg\,max}
\DeclareMathOperator*{\argmin}{arg\,min}
\DeclareMathOperator{\im}{im}
\DeclareMathOperator{\id}{id}
\renewcommand{\mod}[1]{\ (\mathrm{mod}\ #1)}

% section font style
\sectionfont{\sffamily\Large}
\subsectionfont{\sffamily\normalsize}
\subsubsectionfont{\bf}

% line spreading and break
\hyphenpenalty=5000
\tolerance=20
\setlength{\parindent}{0em}
\setlength\parskip{0.5em}
\allowdisplaybreaks
\linespread{0.9}

% enumerate settings
% no break before enumerate
\setlist[enumerate]{itemsep=2pt}

% theorem
% latex theorem
% definition style
\theoremstyle{definition}
\newtheorem{theorem}{Theorem}[subsection]
\newtheorem{axiom}{Axiom}[section]
\newtheorem{definition}{Definition}[section]
\newtheorem{example}{Example}[section]
\newtheorem{question}{Question}[section]
\newtheorem{exercise}{Exercise}[section]
\newtheorem*{exercise*}{Exercise}
\newtheorem{lemma}{Lemma}[section]
\newtheorem{proposition}{Proposition}[section]
\newtheorem{corollary}{Corollary}[section]
\newtheorem*{theorem*}{Theorem}
\newtheorem{problem}{Problem}
% remark style
\theoremstyle{remark}
\newtheorem*{remark}{Remark}
\newtheorem*{solution}{Solution}
\newtheorem*{claim}{Claim}


% paragraph indent
\setlength{\parindent}{0em}
\setlength\parskip{0.5em}

\newcommand\Code{MAT4220 FA22}
\newcommand\Ass{HW05}
\newcommand\name{Haoran Sun}
\newcommand\mail{haoransun@link.cuhk.edu.cn}

\title{{\sffamily \Code \ \Ass}}
\author{\sffamily \name \ (\href{mailto:\mail}{\mail})}
\date{\sffamily \today}

\makeatletter
% \let\Title\@title
\let\theauthor\@author
\let\thedate\@date

\fancypagestyle{plain}{%
    \fancyhf{}
    \lhead{\sffamily \Ass}
    \rhead{\sffamily \name}
    \rfoot{\sffamily\thepage}

    % # 页脚自定义
    \fancyfoot[L]{
        \begin{minipage}[c]{0.06\textwidth}
            \includegraphics[height=7.5mm]{logo2.png}
        \end{minipage}
    }
}
\fancypagestyle{title}{%
    \fancyhf{}
    \renewcommand{\headrulewidth}{0pt}
    % \lhead{\Title}
    % \rhead{\theauthor}
    \rfoot{\sffamily\thepage}

    % # 页脚自定义
    \fancyfoot[L]{
        \begin{minipage}[c]{0.06\textwidth}
            \includegraphics[height=7.5mm]{logo2.png}
        \end{minipage}
    }
}
\fancyfootoffset[L]{0.3cm}

% re-define title format
\makeatletter
\renewcommand{\maketitle}{\bgroup\setlength{\parindent}{0pt}
\begin{flushleft}
  \textbf{\Large\@title}

  \@author
\end{flushleft}\egroup
}
\makeatother

\pagestyle{plain}

% lstlisting settings
\lstset{
    basicstyle=\linespread{0.7}\footnotesize,
    breaklines=true,
    basewidth=0.5em
}


\begin{document}
\maketitle
\thispagestyle{title}
% \begin{multicols*}{2}

% \begin{remark}
%     $V_\epsilon(x)$ is used to denote a $\epsilon$-neighborhood
%     \begin{align*}
%         V_\epsilon(x) = B_\epsilon(x)\setminus\{x\}
%     \end{align*}
% \end{remark}


\begin{problem}[P117 Q4]\
\begin{enumerate}[label=(\alph*)]
\item Note that the formula for cosine coefficients is
\begin{align*}
    A_n &= \frac{1}{l}\int_{-l}^l \phi(x)\cos\frac{n\pi x}{l}\d x
\end{align*}
Since $\phi(x)$ is odd and $\cos\frac{n\pi x}{l}$ is even, we have
$\phi(x)\cos\frac{n\pi x}{l}$ an odd function.
Hence $A_n=0$.

\item Note that the formula for sine coefficients is
\begin{align*}
    B_n &= \frac{1}{l}\int_{-l}^l \phi(x)\sin\frac{n\pi x}{l}\d x
\end{align*}
Since $\phi(x)$ even and $\sin\frac{n\pi x}{l}$ odd, we have
$\phi(x)\sin\frac{n\pi x}{l}$ odd.
Hence $B_n=0$.

\end{enumerate}
\end{problem}


\begin{problem}[P117 Q8]\
\begin{enumerate}[label=(\alph*)]
\item Suppose $f$ is even and differentiable on $(-l, l)$.
Then 
\begin{align*}
    f'(x_0) &= 
    \lim_{h\rightarrow 0}\frac{f(x_0+h) - f(x_0)}{h}
    = \lim_{h\rightarrow 0}-\frac{f(-x_0) - f(-x_0 - h)}{h}
    = -f'(-x_0)
\end{align*}
Then $f'$ is odd.

Following similar steps, we can also prove that $f'$ is even given that $f$ is odd.

\item Suppose $f$ even and integrable on $(-l, l)$.
Then
\begin{align*}
    F(x) &= \int_0^x f(t)\d t
    = \int_0^{-x} -f(-t)\d (-t)
    = -F(-x)
\end{align*}
Then $F$ is odd.

Following similar steps, we can also prove that $F$ is even given that $f$ is odd.

\end{enumerate}
\end{problem}


\begin{problem}[P117 Q10]\
\begin{enumerate}[label=(\alph*)]
\item $\lim_{x\rightarrow 0^+} \phi(x) = 0$

\item $\lim_{x\rightarrow 0^+} \phi(x) = 0$ and $\lim_{x\rightarrow 0^+}\phi'(x)$ exists.

\item $\lim_{x\rightarrow 0^+} \phi(x)$ exists.

\item $\lim_{x\rightarrow 0^+} \phi(x)$ exists and $\lim_{x\rightarrow 0^+}\phi'(x)=0$.


\end{enumerate}
\end{problem}


\begin{problem}[P123 Q10]\
\begin{enumerate}[label=(\alph*)]
\item \begin{proof}
Prove by induction.
Easy to prove that $(Z_1, Z_2)=0$.
Suppose $Z_1,\dots,Z_k$ are orthogonal to each other.
Then 
\begin{align*}
    Y_{k+1} &= X_{k+1} - \sum_{i=1}^k (Z_i, X_{K+1})Z_i\\
    (Z_j, Y_{k+1}) &= (Z_j, Y_{k+1}) - \sum_{k=1}^k (Z_i, X_{k+1})\delta_{ij}
    = (Z_j, Y_{k+1}) - (Z_j, y_{k+1}) = 0
\end{align*}
Then $Y_{k+1}$ is orthogonal to $Z_1,\dots,Z_k$.
Suppose $\|Y_{k+1}\|\neq 0$, then $Z_{k+1}$ exists and
$Z_1,\dots,Z_{k+1}$ orthogonal to each other.
\end{proof}

\item Let $f_1(x) = \cos x + \cos 2x$ and $f_2(x) = 3\cos x - 4\cos 2x$, then
\begin{align*}
    \|f_1\|^2 &= 
    \int_0^\pi \cos^2 x + 2\cos x\cos 2x + \cos^2 2x\d x
    = \pi\\
    \Rightarrow z_1 &= \frac{1}{\sqrt{\pi}}(\cos x + \cos 2x)\\
    y_2 &= f_2 - \frac{1}{\sqrt{\pi}}\int_0^\pi 
    (\cos x + \cos 2x)(3\cos x - 4\cos 2x)\d x
    = \frac{7}{2}(\cos x - \cos 2x)\\
    \Rightarrow z_2 &= \frac{1}{\sqrt{\pi}}(\cos x - \cos 2x)
\end{align*}

\end{enumerate}
\end{problem}


\begin{problem}[P134 Q1]
\begin{remark}
$\forall x\in(-1, 1)$
\begin{align*}
    \lim_{n\rightarrow \infty} S_n = 
    \lim_{n\rightarrow \infty} \sum_{i=0}^n (-1)^n x^{2n} = 
    \frac{1}{1+x^2}
\end{align*}
\end{remark}
\begin{enumerate}[label=(\alph*)]
\item Note that
\begin{align*}
    |S_n - S| &= \left|\frac{- (-x^2)^n}{1+x^2}\right|
    = \frac{x^{2n}}{1 + x^2}
\end{align*}
Easy to show that for each $x\in (-1, 1)$, $\forall\epsilon>0$, we can choose 
$N\in\N$ where $x^{2N}<\epsilon$
s.t. $|S_n-S|<\epsilon$ $\forall n>N$.
Then $S_n$ converges pointwisely.

\item $S_n$ does not converges uniformly since there always $\exists\epsilon=1/8>0$,
s.t. $\forall n>N\in\N$, we can always find $x_0 = (1/2)^{1/2n}$ where
\begin{align*}
    |S_n(x_0) - S(x_0)| &=
    \frac{x^{2n}}{1 + x^2} > \frac{1}{2}x^{2n} 
    = \frac{1}{4} > \epsilon = \frac{1}{8}
\end{align*}

\item $S_n$ converges in the $L^2$ sense since $\forall\epsilon>0$, we can choose
$N\in\N$ where $N>1/\epsilon$ s.t. 
\begin{align*}
    \left|\int_{-1}^1 \left(
        \frac{x^{2n}}{1+x^2}
    \right)^2\d x\right|
    &\leq \int_{-1}^1 \frac{1}{4}x^{2n} \d x
    = \frac{1}{2}\frac{1}{2n+1} < \frac{\epsilon }{2+\epsilon } < \epsilon
\end{align*}


\end{enumerate}
\end{problem}


\begin{problem}[P134 Q3]\
\begin{enumerate}[label=(\alph*)]
\item $\forall x\notin [1/2-1/n, 1/2)\cup(1/2, 1/2+1/n]$, we have $f_n(x)=0$.
Forall other $x$, $\forall \epsilon>0$, $\exists N\in\N$ with $1/N<|x-\frac{1}{2}|$ 
s.t. $\forall n>N$, we have $f_n(x) = 0$.
Then $f_n\rightarrow 0$ pointwisely.

\item $\exists\epsilon=1$, $\exists N_1\in\N$ s.t. $|\gamma_n|>\epsilon=1$ $\forall n>N_1$
since $\gamma_n\rightarrow\infty$.
Then $\forall n > N_1$, $\exists x_n = x+1/2n$ s.t. $|f_n(x)|=|\gamma_n|>\epsilon=1$.
Then $f_n\rightarrow 0$ not uniformly.

\item Easy to show that
\begin{align*}
    \|f_n\|^2 &= \frac{2}{n}\cdot n^{2/3} = 2n^{-1/3}\rightarrow 0
    \text{ as } n\rightarrow \infty
\end{align*}
$f_n\rightarrow 0$ in $L^2$ sense.

\item Easy to show that
\begin{align*}
    \|f_n\|^2 &= \frac{2}{n}n^2 = 2n \rightarrow \infty
\end{align*}
$f_n\rightarrow 0$ not in $L^2$ sense.


\end{enumerate}
\end{problem}


\begin{problem}[P134 Q7]\
\begin{enumerate}[label=(\alph*)]
\item We can verify that
\begin{align*}
    c_0 &= \frac{1}{2}\int_{-1}^1\phi(x)\d x = 0\\
    c_n &= \frac{1}{2}\int_{-1}^1e^{-in\pi x}\phi(x)
    = \frac{1}{2}\left[
        \int_{-1}^0 e^{-in\pi x}(-1-x)\d x
        + \int_0^1 e^{-in\pi x}(1-x)\d x
    \right] 
    = \frac{1}{in\pi}
\end{align*}

\item First non-zero term (traditional sine series) is
\begin{align*}
    \frac{2}{\pi}\sin  \pi x,\
    \frac{1}{\pi}\sin 2\pi x,\
    \frac{2}{3\pi}\sin 3\pi x
\end{align*}

\item Note that
\begin{align*}
    \|\phi\|^2 &= \int_{-1}^1\phi(x)^2\d x = \frac{2}{3}\\
    \|S_N^2\| &= \sum_{n=-N}^N |c_n|^2 \int_{-1}^1 1\d x
    = \frac{4}{\pi^2}\sum_{n=1}^N \frac{1}{n^2} \rightarrow \frac{2}{3}
\end{align*}
Then $\|S_n\|$ not converge to $\|\phi\|$, then
it converges in the $L^2$ sense.

\item It converges pointwisely.

\item It does not converges uniformly.


\end{enumerate}
\end{problem}







% \end{multicols*}
\end{document}

