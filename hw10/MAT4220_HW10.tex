\documentclass[twoside,11pt]{article}
\usepackage[left=1in, right=1in, top=1in, bottom=1in]{geometry}
\usepackage{amsmath}
\usepackage{amssymb}
\usepackage{amsfonts}
\usepackage{mathtools}
\usepackage{amsthm}
\usepackage{fancyhdr}
\usepackage{enumitem}
\usepackage{siunitx}
\usepackage{booktabs}
\usepackage[hidelinks]{hyperref}
\usepackage{sectsty}
\usepackage{mathrsfs} % mathscr
\usepackage{tikz}
\usepackage{pgfplots}
\usepackage{multicol}
\usepackage{listings}
\usepackage{soul}
\usepackage{braket}
\usepackage{esint}
% \usepackage{amsart}

% lmodern
\usepackage{lmodern}

% change mathcal shape
\usepackage[mathcal]{eucal}

% allow H option of figure
\usepackage{float}

% define math operators
\newcommand{\F}{\mathbb{F}}
\newcommand{\R}{\mathbb{R}}
\newcommand{\N}{\mathbb{N}}
\newcommand{\Z}{\mathbb{Z}}
\newcommand{\Q}{\mathbb{Q}}
\newcommand{\X}{\mathbb{Y}}
\renewcommand{\L}{\mathcal{L}}
% \renewcommand{\d}{\mathrm{d}}
\renewcommand*\d{\mathop{}\!\mathrm{d}}
\DeclareMathOperator*{\argmax}{arg\,max}
\DeclareMathOperator*{\argmin}{arg\,min}
\DeclareMathOperator{\im}{im}
\DeclareMathOperator{\id}{id}
\renewcommand{\mod}[1]{\ (\mathrm{mod}\ #1)}

% section font style
\sectionfont{\sffamily\Large}
\subsectionfont{\sffamily\normalsize}
\subsubsectionfont{\bf}

% line spreading and break
\hyphenpenalty=5000
\tolerance=20
\setlength{\parindent}{0em}
\setlength\parskip{0.5em}
\allowdisplaybreaks
\linespread{0.9}

% enumerate settings
% no break before enumerate
\setlist[enumerate]{itemsep=2pt,topsep=2pt}

% theorem
% latex theorem
% definition style
\theoremstyle{definition}
\newtheorem{theorem}{Theorem}[subsection]
\newtheorem{axiom}{Axiom}[section]
\newtheorem{definition}{Definition}[section]
\newtheorem{example}{Example}[section]
\newtheorem{question}{Question}[section]
\newtheorem{exercise}{Exercise}[section]
\newtheorem*{exercise*}{Exercise}
\newtheorem{lemma}{Lemma}[section]
\newtheorem{proposition}{Proposition}[section]
\newtheorem{corollary}{Corollary}[section]
\newtheorem*{theorem*}{Theorem}
\newtheorem{problem}{Problem}
% remark style
\theoremstyle{remark}
\newtheorem*{remark}{Remark}
\newtheorem*{solution}{Solution}
\newtheorem*{claim}{Claim}


% paragraph indent
\setlength{\parindent}{0em}
\setlength\parskip{0.5em}

\newcommand\Code{MAT4220 FA22}
\newcommand\Ass{HW10}
\newcommand\name{Haoran Sun}
\newcommand\mail{haoransun@link.cuhk.edu.cn}

\title{{\sffamily \Code \ \Ass}}
\author{\sffamily \name \ (\href{mailto:\mail}{\mail})}
\date{\sffamily \today}

\makeatletter
% \let\Title\@title
\let\theauthor\@author
\let\thedate\@date

\fancypagestyle{plain}{%
    \fancyhf{}
    \lhead{\sffamily \Ass}
    \rhead{\sffamily \name}
    \rfoot{\sffamily\thepage}

    % # 页脚自定义
    \fancyfoot[L]{
        \begin{minipage}[c]{0.06\textwidth}
            \includegraphics[height=7.5mm]{logo2.png}
        \end{minipage}
    }
}
\fancypagestyle{title}{%
    \fancyhf{}
    \renewcommand{\headrulewidth}{0pt}
    % \lhead{\Title}
    % \rhead{\theauthor}
    \rfoot{\sffamily\thepage}

    % # 页脚自定义
    \fancyfoot[L]{
        \begin{minipage}[c]{0.06\textwidth}
            \includegraphics[height=7.5mm]{logo2.png}
        \end{minipage}
    }
}
\fancyfootoffset[L]{0.3cm}

% re-define title format
\makeatletter
\renewcommand{\maketitle}{\bgroup\setlength{\parindent}{0pt}
\begin{flushleft}
  \textbf{\Large\@title}

  \@author
\end{flushleft}\egroup
}
\makeatother

\pagestyle{plain}

% lstlisting settings
\lstset{
    basicstyle=\linespread{0.7}\footnotesize,
    breaklines=true,
    basewidth=0.5em
}


\begin{document}
\maketitle
\thispagestyle{title}
% \begin{multicols*}{2}

% \begin{remark}
%     $V_\epsilon(x)$ is used to denote a $\epsilon$-neighborhood
%     \begin{align*}
%         V_\epsilon(x) = B_\epsilon(x)\setminus\{x\}
%     \end{align*}
% \end{remark}

\begin{problem}[P352 Q1]
Let 
\begin{align*}
    \tilde{u}(\xi, t) &= 
    \int\d x \ e^{-i\xi x} u(x, t)
\end{align*}
Hence
\begin{align*}
    \tilde{u}_t(\xi, t) &= -\kappa\xi^2\tilde{u}(\xi, t)
    + i\mu\xi\tilde{u}(\xi, t)
    \Rightarrow
    \tilde{u}(\xi, t) =
    e^{-\kappa\xi^2 t + i\mu\xi t}C(\xi)
\end{align*}
which means
\begin{align*}
    u(x, t) &= 
    \frac{1}{2\pi}\int\d\xi\ e^{-\kappa\xi^2 t + i\mu\xi t} e^{i\xi x}C(\xi)\\
    u(x, 0) &= 
    \frac{1}{2\pi}\int\d\xi\ e^{i\xi x}C(\xi) = \phi(x)\\
    \Rightarrow C(\xi) &= \int\d x\ \phi(x)e^{-i\xi x}
\end{align*}
then we solved the equation.

\end{problem}


\begin{problem}[P352 Q2]
Let
\begin{align*}
    \tilde{u}(\mu, y) &= 
    \int\d x\ e^{-i\mu x}u(x, y)
\end{align*}
Hence
\begin{align*}
    \frac{\partial^2}{\partial y^2}\tilde{u}(\mu, y) &= 
    \mu^2\tilde{u}(\mu, y)\Rightarrow
    \tilde{u}(\mu, y) = A(\mu)\sinh\mu y + B(\mu)\cosh\mu y
\end{align*}
which means
\begin{align*}
    u(x, y) &= \frac{1}{2\pi}\int\d\mu\ e^{i\mu x}[A(\mu)\sinh\mu y + B(\mu)\cosh\mu y]\\
    u_y(x, y) &= \frac{1}{2\pi}\int\d\mu\ e^{i\mu x}[\mu A(\mu)\cosh\mu y + \mu B(\mu)\sinh\mu y]\\
    u_y(x, 0) &= \frac{1}{2\pi}\int\d\mu\ e^{i\mu x}\mu A(\mu) = h(x)\\
    \Rightarrow A(\mu) &= \frac{1}{\mu}\int\d x\ e^{-i\mu x}h(x)
\end{align*}
then we solved the equation.


\end{problem}





% \end{multicols*}
\end{document}

